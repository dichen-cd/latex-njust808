% !Mode:: "TeX:UTF-8"
% 这个是为了WinEdt设置的,它的默认不是UTF8.
% !TeX root = ../main.tex
% 若xelatex编译非UTF8文件,需在每个文件中指定字符编码;
% main.tex中手动制定了\atemp和\usewhat参数;
\ifx\atempxetex\usewhat
%\XeTeXinputencoding "gbk"
\fi
\defaultfont

\thispagestyle{empty}
%在此节中,由于用了与其它正文不同的字体,此节中用的是
%四号字体,所以要对行距与首行缩进重新定义。
\begin{center}{\sanhao \hei{声\quad 明}}\end{center}

\renewcommand{\baselinestretch}{1.5}\large{}
{\setlength{\parindent}{2em}本学位论文是我在导师的指导下取得的研究成果,尽我所知,在本
 学位论文中,除了加以标注和致谢的部分外,不包含其他人已经发表或
公布过的研究成果,也不包含我为获得任何教育机构的学位或学历而使
用过的材料。与我一同工作的同事对本学位论文做出的贡献均已在论文
中作了明确的说明。}

    \vspace{0.738cm}
    \begin{flushleft}{
    研究生签名:\underline{~~~~~~~~~~~~~~~}~~~~~~~~~~~~~~~~日期:~~~~~~~~~~~年~~~~~月~~~~~日}
    \end{flushleft}

    \vspace{2.214cm}
%%%%%%%%%%%%%%%%%%南京理工大学博(硕)士学位论文使用授权书%%%%%%%%%%%%%%%%%%%
%\phantomsection
    \begin{center}{\sanhao \hei{南京理工大学\cxuewei 学位论文使用授权声明}}
    \end{center}

    \vspace{0.738cm}

\renewcommand{\baselinestretch}{1.5}\large{}
{\setlength{\parindent}{2em}南京理工大学有权保存本学位论文的电子和纸质文档,可以借阅或上网公布本学位论文的部分或全部内容,
可以向有关部门或机构送交并授权其保存、借阅或上网公布本学位论文的部分或全部内容。对于保密论文,
按保密的有关规定和程序处理。}


\vspace{1.476cm}
\begin{flushleft}{
研究生签名:\underline{~~~~~~~~~~~~~~~~~}~~~~~~~~~~~~~~日期:~~~~~~~~~~~年~~~~~月~~~~~日}
\end{flushleft}
\vspace{0.2cm}
\iffalse
\begin{flushleft}{
\hspace*{8pt}导师签名:\underline{~~~~~~~~~~~~~~~~~~~}~~~~~~~~~~~~~~日期:~~~~~~~~~~~年~~~~~月~~~~~日}
\end{flushleft}
\fi
\newpage
