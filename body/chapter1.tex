% !Mode:: "TeX:UTF-8"
% 这个是为了WinEdt设置的,它的默认不是UTF8.
% !TeX root = ../main.tex
% 这个为TexWorks设置,可以用来编译多个文件。

% 若xelatex编译非UTF8文件,需在每个文件中指定字符编码;
% main.tex中手动制定了\atemp和\usewhat参数;
\ifx\atempxetex\usewhat
%\XeTeXinputencoding "gbk"
\fi

\defaultfont

\titleformat{\chapter}[hang]{\xiaosan\bf\raggedright\song\sf\boldmath}{\xiaoer\chaptertitlename}{18pt}{\xiaosan}
\titlespacing{\chapter}{0pt}{8pt}{16pt}

\makeatletter
\newskip\@footindent
\@footindent=1em

\renewcommand\footnoterule{\kern-3\p@ \hrule width 0.4\columnwidth \kern 2.6\p@}
\@addtoreset{footnote}{page}

\long\def\@makefntext#1{\@setpar{\@@par\@tempdima \hsize
\advance\@tempdima-\@footindent
\parshape \@ne \@footindent \@tempdima}\par
\noindent \hbox to \z@{\hss\@thefnmark\hspace{0.5em}}#1}

\renewcommand\thefootnote{\pinumber{\arabic{footnote}}}
\def\@makefnmark{\hbox{\textsuperscript{\@thefnmark}}}

\newcommand\pinumber[1]{\ifcase#1 \or \ding{172}\or \ding{173}\or
  \ding{174}\or \ding{175}\or \ding{176}\or \ding{177}%
  \or \ding{178}\or \ding{179}\or \ding{180}\or \ding{181}\else *\fi\relax}
\makeatother
%以上从\makeatletter到\makeatother为重定义脚注编号,使之带圆圈

\chapter{绪论}
\label{cha1:introduction}


\section{研究背景}
\label{sec1:background}

\subsection{图论的兴起}
\label{subsec1:graph_theory}



随着科技的迅猛发展,越来越多的问题可以转化成为图,基于图的研究越来越多。
\cite{Lenard2011}




\subsection{当前数据的复杂性}
\label{subsec1:data_complexity}

\section{研究意义}
\label{sec1:motivation}


\section{本文的主要贡献}
\label{sec1:contribution}


\section{本文的组织结构}
\label{sec1:organization}

本文的主要结构如下。

第一章绪论,简要介绍本文研究工作的背景及意义,以及研究工作的意义,以及本文的主要贡献。

第二章文献综述。

最后结论,总结全文的主要工作,以及对于未来工作的展望。

