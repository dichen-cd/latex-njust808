% !Mode:: "TeX:UTF-8"
% The main file for the thesis.
% main.tex
% 

% 编译方式,可选的有xelatex, pdflatex, dvipdfmx, dvipspdf, yap
% 推荐使用xelatex,默认UTF8编码格式,对中文支持很好。
\def\usewhat{xelatex} 

% 若用xelatex编译,默认用UTF8编码,非UTF8文件时,需在每个文件中指定字符编码;
% 本段无需改动,需放在输入中文之前;
\def\atempxetex{xelatex}\ifx\atempxetex\usewhat
% 若文件使用非utf8编码,请将注释去掉
% \XeTeXinputencoding "gbk"
\fi

% input "reference\reference.bib" %for winedt users
% 版本号: X.Y.Z, X: major, Y: minor, Z: patch
\def\version{1.0.0}     % 该变量仅用于模板文件的版本号控制

\def\xuewei{Doctor}   % 定义学位 博士
%\def \xuewei {Master}  % 硕士

\def\oneortwoside{twoside} %定义单双面打印,只对硕士学位论文有效;
%\def\oneortwoside{oneside} % 硕士单面打印

\def\xueke{Engineering} % 定义学科 工学
%\def\xueke{Science}      % 理学
%\def\xueke{Management}   % 管理学
%\def\xueke{Arts}         % 艺术学

% dicta 2009 definition.
\DeclareMathAlphabet{\mathpzc}{OT1}{pzc}{m}{it}
\def\calV{\mathcal{V}}
\def\calE{\mathcal{E}}
\def\calC{\mathcal{C}}
\def\calU{\mathcal{U}}
\def\bfX{\mathbf{X}}
\def\bfx{\mathbf{x}}
\def\bfY{\mathbf{Y}}
\def\bfy{\mathbf{y}}
\def\bfL{\mathbf{L}}
\def\bfE{\mathbf{E}}
\def\bfD{\mathbf{D}}
\def\bfW{\mathbf{W}}
\def\bfT{\mathpzc{T}}
\def\bfone{\mathbf{1}}
\def\overX{\overline{X}}
\def\vector{\mathrm{vec}}
\def\lsep{\!\!&}
\def\rsep{&\!\!}
% end of dicta 2009

%%%%%%%%%%%%%%%%%%%%%%%%%%%cviu2009special%%%%%%
\def\calR{\mathcal{R}}
\def\boldx{\mathbf{x}}
\def\boldX{\mathbf{X}}
\def\boldS{\mathbf{S}}
\def\boldm{\mathbf{m}}
\def\boldZ{\mathbf{Z}}
\def\boldI{\mathbf{I}}
%\def\boldY{\mathbf{Y}}
\def\ddd{\mathbf{\nabla}}
\def\boldQ{\mathbf{Q}}
\def\boldR{\mathbf{R}}
\def\boldu{\mathbf{u}}
\DeclareMathAlphabet{\mathpzc}{OT1}{pzc}{m}{it}
%%%%%%%%%%%%%%%%%%%%%%%%%%%%%%%%%%%%%%%%%%%%%%%%%%%%5

% !Mode:: "TeX:UTF-8"
% !TeX root = ../main.tex
% 这个是为了WinEdt设置的,它的默认不是UTF8.
% 若xelatex编译非UTF8文件,需在每个文件中指定字符编码;
% main.tex中手动制定了\atemp和\usewhat参数;
\ifx\atempxetex\usewhat
%\XeTeXinputencoding "gbk"
\fi

% 硕博类型 的一些定义

% 导言区使用中文
\makeatletter
\@tempcnta=128
\loop \catcode\@tempcnta=13 \ifnum\@tempcnta<255 \advance \@tempcnta \@ne
\repeat
\makeatother

\newif\ifxueweidoctor %判断论文类型
\newif\ifxueweimaster
\def\temp{Doctor}
\ifx\temp\xuewei
  \xueweidoctortrue  \xueweimasterfalse
\fi
\def\temp{Master}
\ifx\temp\xuewei
  \xueweidoctorfalse  \xueweimastertrue
\fi

\ifxueweidoctor
  \newcommand{\cxuewei}{博士}
  \newcommand{\exuewei}{Doctor}
  \newcommand{\exueweier}{Doctoral}
  \newcommand{\xueweishort}{博}
\fi

\ifxueweimaster
  \newcommand{\cxuewei}{硕士}
  \newcommand{\exuewei}{Master}
  \newcommand{\exueweier}{Master}
  \newcommand{\xueweishort}{硕}
\fi


\ifxueweidoctor
  \def\oneortwoside{twoside}
\fi

\ifx\oneortwoside\undefined
  \def\oneortwoside{twoside}
\fi

\newif\ifoneortwoside
\def\temp{twoside}
\ifx\temp\oneortwoside
  \oneortwosidetrue
\else
  \oneortwosidefalse
\fi
    % 硕博类型

%下面的book选项中可以使用 draft 选项,使插入的图形只显示外框,以加快预览速度。
\documentclass[12pt,a4paper,openany,\oneortwoside]{book}
% !Mode:: "TeX:UTF-8"
% !TeX root = ../main.tex
% 这个是为了WinEdt设置的,它的默认不是UTF8.
% 若xelatex编译非UTF8文件,需在每个文件中指定字符编码;
% main.tex中手动制定了\atemp和\usewhat参数;
\ifx\atempxetex\usewhat
%\XeTeXinputencoding "gbk"
\fi

\usepackage{amssymb}
\usepackage{amsmath}        % AMSLaTeX宏包 用来排出更加漂亮的公式

% 图形支持宏包 为了使用pdftex 需要作相应判断
\usepackage{etex}%增加计数器总数(原来是256,宏包多,可能不够用),编译需基于eTeX
\usepackage{ifpdf}
%定义一个新判断命令 %有一个宏包ifpdf 可以完成这件事,应该比这个严谨
%\newif\ifpdf
%\ifx\pdfoutput\undefined
%   \pdffalse
%\else
%   \pdfoutput=1
%   \pdftrue
%\fi
\usepackage{pifont}%调用文本个带圈的宏包
%%%%%%%%彩色引用和书签%%%%%%
\ifx\atempxetex\usewhat
    \usepackage[dvipdfm]{graphicx}
\else
    \ifpdf
        \usepackage[pdftex]{graphicx}
    \else
        \usepackage[dvips]{graphicx}
    \fi
\fi
\usepackage[top=30mm,bottom=24mm,left=25mm,right=25mm]{geometry}
\usepackage{layouts}                    % 打印当前页面格式的宏包
\usepackage[sf]{titlesec}               % 控制标题的宏包
\usepackage{titletoc}                   % 控制目录的宏包
\usepackage[perpage,symbol]{footmisc}   % 脚注控制
\usepackage{fancyhdr}                   % fancyhdr宏包 页眉和页脚的相关定义
\usepackage{fancyref}
\usepackage{array}          %增强表格的功能 %可能与xeCJK宏包冲突,需放在xeCJK之前。

\ifx\atempxetex\usewhat
\usepackage[slantfont,boldfont,CJKaddspaces]{xeCJK} %
%\CJKlanguage{zh-cn}
\else
\usepackage{CJK,CJKpunct}   % 中文支持宏包
\usepackage{times}          % 使用Times字体的宏包
\fi

\usepackage{type1cm}        % tex1cm宏包,控制字体的大小
\usepackage{indentfirst}    % 首行缩进宏包
\usepackage{color}          % 支持彩色

\usepackage{relsize}            % 调整公式字体大小 \mathsmaller \mathlarger
\usepackage{textcomp}         % 千分号等特殊符号
\usepackage{mathrsfs}       % 不同于\mathcal or \mathfrak 之类的英文花体字体
\usepackage{bm}              % 处理数学公式中的黑斜体的宏包
\usepackage[amsmath,thmmarks,hyperref]{ntheorem}% 定理类环境宏包,其中 amsmath 选项用来兼容 AMS LaTeX 的宏包

\usepackage{epsfig}         % eps图像
\usepackage[below]{placeins}%允许上一个section的浮动图形出现在下一个section的开始部分,还提供\FloatBarrier命令,使所有未处理的浮动图形立即被处理
%\usepackage{psfrag}        %替换eps图形中的文字
%\usepackage{floatflt}       % 图文混排用宏包
\usepackage{rotating}       % 图形和表格的控制
%\usepackage{endfloat}      %可将浮动对象放置到文件的最后
\usepackage{setspace}       % 定制表格和图形的多行标题行距
\usepackage{flafter}       % 使得所有浮动体不能被放置在其浮动环境之前,以免浮动体在引述它的文本之前出现.
\usepackage{multirow}       %使用Multirow宏包,使得表格可以合并多个row格
\usepackage{booktabs}       % 表格,横的粗线;\specialrule{1pt}{0pt}{0pt}
\usepackage{longtable}      %支持跨页的表格。
%\usepackage[centerlast]{caption2}       %浮动图形和表格标题样式,这里已被ccaption完全替代。
\usepackage[hang,center]{subfigure}%支持子图 %centerlast 设置最后一行是否居中
\usepackage[subfigure]{ccaption} %,caption2,

%\usepackage{cite}          % 支持引用的宏包
\usepackage[sort&compress,numbers]{natbib}% 支持引用缩写的宏包
\usepackage{hypernat}

\usepackage{enumitem}       %使用enumitem宏包,改变列表项的格式
\usepackage{calc}           %长度可以用+ - * / 进行计算
\usepackage[boxed,linesnumbered,algochapter]{algorithm2e}    % 算法的宏包

% 生成有书签的pdf及其开关, 该宏包应放在所有宏包的最后, 宏包之间有冲突
\def\atemp{dvipspdf}\ifx\atemp\usewhat
\usepackage[dvips,
            CJKbookmarks=true,
            bookmarksnumbered=true,
            bookmarksopen=true,
            colorlinks=false,    % 最后打印的时候可以改成false,这样字体都是黑色
            pdfborder={0 0 1},   % 去掉链接的边框
            citecolor=blue,
            linkcolor=red,
            anchorcolor=green,
            urlcolor=blue,
            unicode,
            breaklinks=true
            ]{hyperref}
\usepackage{breakurl} %消除dvips的时候网页链接断行失效的问题。
\fi

\def\atemp{dvipdfmx}\ifx\atemp\usewhat
\usepackage[dvipdfm, %dvi-->pdf 生成书签
            CJKbookmarks=true,
            bookmarksnumbered=true,
            bookmarksopen=true,
            colorlinks=false,
            pdfborder={0 0 1},
            citecolor=blue,
            linkcolor=red,
            anchorcolor=green,
            urlcolor=blue,
            breaklinks=true
            ]{hyperref}
\AtBeginDvi{\special{pdf:tounicode GBK-EUC-UCS2}} % GBK -> Unicode,此时可以不用gbk2uni
\fi

\def\atemp{pdflatex}\ifx\atemp\usewhat
\usepackage{cmap}                       %pdflatex编译时,可以生成可复制、粘贴的中文PDF文档
\usepackage[pdftex,
            CJKbookmarks=true,
            bookmarksnumbered=true,
            bookmarksopen=true,
            colorlinks=false,
            pdfborder={0 0 1},
            citecolor=blue,
            linkcolor=red,
            anchorcolor=green,
            urlcolor=blue,
            unicode,
            breaklinks=true
            ]{hyperref}
\fi

\def\atemp{yap}\ifx\atemp\usewhat
\usepackage[dvipdf,  %考虑到yap可以正反向搜索,dvipdf使yap中的链接有效。
            CJKbookmarks=true,
            bookmarksnumbered=true,
            bookmarksopen=true,
            colorlinks=false,
            pdfborder={0 0 1},
            citecolor=blue,
            linkcolor=red,
            anchorcolor=green,
            urlcolor=blue,
            unicode,
            breaklinks=true
            ]{hyperref}
\fi

\ifx\atempxetex\usewhat %\def\atempxetex{xelatex} main.tex中已定义;
\usepackage[xetex,
            CJKbookmarks=true,
            bookmarksnumbered=true,
            bookmarksopen=true,
            colorlinks=false,
            pdfborder={0 0 1},
            citecolor=blue,
            linkcolor=red,
            anchorcolor=green,
            urlcolor=blue,
            breaklinks=true,
            naturalnames  %与algorithm2e宏包协调
            ]{hyperref}
\fi

\usepackage{arydshln}       %分块矩阵画虚线,挺好用
 % 引用的宏包


% 论文包含的内容
\includeonly{
                appendix/Authorization,
                body/chapter1,
                body/chapter2,
                body/chapter3,
                body/chapter4,
                body/chapter5,
                body/conclusion,
                body/acknowledgements,
                appendix/appA,
                appendix/publications,
                appendix/Resume
            }
\graphicspath{{figures/}} % 定义所有的eps文件在 figures 子目录下

\begin{document}
\ifx\atempxetex\usewhat\else
\begin{CJK*}{UTF8}{song}
\fi

\input{setup/Definition} % 文本格式定义
% !Mode:: "TeX:UTF-8"
% !TeX root = ../main.tex
% 这个是为了WinEdt设置的,它的默认不是UTF8.
% 若xelatex编译非UTF8文件,需在每个文件中指定字符编码;
% main.tex中手动制定了\atemp和\usewhat参数;
\ifx\atempxetex\usewhat
%\XeTeXinputencoding "gbk"
\fi

%%%%%%%%%%%%%%%%%%%%%%%%%%%%%%%%%%%%%%%%%%%%%%%%%%%%%%%%%%%
%允许公式换页显示,否则大型推导公式都在一页内,
%一页显示不下放到第二页,导致很大的空白空间,很不好看
\allowdisplaybreaks[4]

%%%%%%%%%%%%%%%%%%%%%%%%%%%%%%%%%%%%%%%%%%%%%%%%%%%%%%%%%%%
%下面这组命令使浮动对象的缺省值稍微宽松一点,从而防止幅度
%对象占据过多的文本页面,也可以防止在很大空白的浮动页上放置很小的图形。
\renewcommand{\topfraction}{0.9999999}
\renewcommand{\textfraction}{0.0000001}
\renewcommand{\floatpagefraction}{0.9999}

%%%%%%%%%%%%%%%%%%%%%%%%%%%%%%%%%%%%%%%%%%%%%%%%%%%%%%%%%%%
% 重定义一些正文相关标题
%%%%%%%%%%%%%%%%%%%%%%%%%%%%%%%%%%%%%%%%%%%%%%%%%%%%%%%%%%%
\theoremstyle{plain} \theorembodyfont{\song\rmfamily}
\theoremheaderfont{\hei\rmfamily} %\theoremseparator{:}
\newtheorem{definition}{\hei 定义}[chapter]
\newtheorem{example}{\hei 例}[chapter]
\newtheorem{algo}{\hei 算法}[chapter]
\newtheorem{theorem}{\hei 定理}[chapter]
\newtheorem{axiom}{\hei 公理}[chapter]
\newtheorem{proposition}{\hei 命题}[chapter]
\newtheorem{lemma}{\hei 引理}[chapter]
\newtheorem{corollary}{\hei 推论}[chapter]
\newtheorem{remark}{\hei 注解}[chapter]
%\newtheorem{proposition}[definition]{\hei 命题}
%\newtheorem{lemma}[definition]{\hei 引理}
%\newtheorem{exercise}[definition]{}
%\newtheorem{corollary}[definition]{\hei 推论}
%\newtheorem{remark}[definition]{\hei 注解}
%%%%%%%%%%%%%%%%%%%%%%%%%%%%%%%%%%%%%%%%%%%%%%%%%%%%%%%%%%%%%%%%%%%
%解决原proof定理环境的两个问题:
%  1. proof 中的item缩进不对
%  2. proof 中的最后一个公式下出现一个黑方块。
%\theoremsymbol{$\blacksquare$}
%\newtheorem{proof}{\hei 证明}
\newenvironment{proof}{\noindent{\hei 证明:}}{\hfill $ \square $ \vskip 4mm}
\theoremsymbol{$\square$}


%%%%%%%%%%%%%%%%%%%%%%%%%%%%%%%%%%%%%%%%%%%%%%%%%%%%%%%%%%%
% 用于中文段落缩进 和正文版式
%%%%%%%%%%%%%%%%%%%%%%%%%%%%%%%%%%%%%%%%%%%%%%%%%%%%%%%%%%%

\ifx\atempxetex\usewhat
\newcommand{\CJKcaption}[1]{
  \ifx\CJK@actualBinding \@empty
    \PackageError{CJK}{
      You must be inside of a CJK environment to use \protect\CJKcaption}{}
  \else
    \makeatletter
    \InputIfFileExists{#1.cpx}{}{
      \PackageError{CJK}{
        Can't find #1.cpx}{
        The default captions are used if you continue.}}
    \makeatother
  \fi}
\CJKcaption{gb_452}
\else
\CJKcaption{gb_452}
\newlength \CJKtwospaces
\def\CJKindent{
    \settowidth\CJKtwospaces{\CJKchar{"0A1}{"0A1}\CJKchar{"0A1}{"0A1}}%
    \parindent\CJKtwospaces
}
\CJKtilde  \CJKindent
\fi
\renewcommand\CJKprechaptername{}
\renewcommand\CJKchaptername{}
\setlength{\parindent}{26pt} %由于工大论文的每行的字距加大了,需要增加段首缩2pt

\renewcommand\contentsname{\hei 目~~~~录}

%%%%%%章节标题为“1”的形式
\renewcommand\chaptername{\arabic{chapter}}
%%%%%%%%%%%%%%%%%%%%%%%%%%%%%%%%%%%%%%%%%%%%%%%%%%
%定义段落章节的标题和目录项的格式
%%%%%%%%%%%%%%%%%%%%%%%%%%%%%%%%%%%%%%%%%%%%%%%%%%
\setcounter{secnumdepth}{4} \setcounter{tocdepth}{2}

\titleformat{\chapter}[hang]{\xiaosan\bf\filcenter\song\sf\boldmath}{\xiaosan\chaptertitlename}{1em}{\xiaosan}
\titlespacing{\chapter}{0pt}{18pt}{18pt}

\titleformat{\section}[hang]{\song\sihao\sf\bf\boldmath}{\song\sihao\sf\bf\thesection}{0.5em}{}
\titlespacing{\section}{0pt}{12pt}{12pt}

\titleformat{\subsection}[hang]{\song\sf\xiaosi\bf\boldmath}{\xiaosi\thesubsection}{0.5em}{}
\titlespacing{\subsection}{0pt}{6pt}{6pt}

\titleformat{\subsubsection}[hang]{\song\sf\xiaosi\boldmath}{\xiaosi\thesubsubsection}{0.5em}{}[\;\;]
\titlespacing{\subsubsection}{0pt}{3pt}{2pt}

% 缩小目录中各级标题之间的缩进,使它们相隔一个字符距离,也就是12pt
\makeatletter
\renewcommand*\l@section{\@dottedtocline{1}{12pt}{18pt}}
\renewcommand*\l@subsection{\@dottedtocline{2}{24pt}{27pt}}
\renewcommand*\l@subsubsection{\@dottedtocline{3}{36pt}{39pt}}
\renewcommand*\l@paragraph{\@dottedtocline{4}{48pt}{48pt}}
\renewcommand*\l@subparagraph{\@dottedtocline{5}{60pt}{60pt}}

%控制中文目录
\dottedcontents{chapter}[3.4em]{\vspace{0.5em}\hspace{-3.4em}\hei \bf\boldmath}{0.0em}{5pt}% 章标题后用粗点
%\titlecontents{chapter}[3.92em]{\vspace{0.5em}\hspace{-3.92em}\hei \bf\boldmath}{\contentslabel{0em}}{\hspace*{-0em}}{\normalfont\titlerule*[5pt]{.}\contentspage} %章标题后用细点
\dottedcontents{section}[1.6cm]{}{1.8em}{5pt}
\dottedcontents{subsection}[2.840cm]{}{2.7em}{5pt}
\dottedcontents{subsubsection}[3.78cm]{}{3.4em}{5pt}

%%%%%%%%%%%%%%%%%%%%%%%%%%%%%%%%%%%%%%%%%%%%%%%%%%%%%%%
% 定义页眉和页脚 使用fancyhdr 宏包
%%%%%%%%%%%%%%%%%%%%%%%%%%%%%%%%%%%%%%%%%%%%%%%%%%%%%%%%
\newcommand{\makeheadrule}{%
\makebox[-3pt][l]{\rule[.7\baselineskip]{\headwidth}{0.4pt}}
\rule[0.85\baselineskip]{\headwidth}{0pt}\vskip-.8\baselineskip}%控制页眉上的线粗细
\renewcommand{\headrule}{%
    {\if@fancyplain\let\headrulewidth\plainheadrulewidth\fi
     \makeheadrule}}

\pagestyle{fancyplain}

%去掉章节标题中的数字
%%不要注销这一行,否则页眉会变成:“第1章1  绪论”样式
\renewcommand{\chaptermark}[1]{\markboth{\chaptertitlename~~ \ #1}{}}
 \fancyhf{}

%在book文件类别下,\leftmark自动存录各章之章名,\rightmark记录节标题
%% 页眉字号 南理工要求小五宋体
%根据单双面打印设置不同的页眉;
\ifoneortwoside
  \fancyhead{} % clear all fields
  \fancyhead[LO]{\CJKfamily{song}\xiaowu{}\cxuewei 论文}
  \fancyhead[RO]{\CJKfamily{song}\xiaowu 基于图的模式识别及其在计算机视觉中的应用}
  \fancyhead[LE]{\CJKfamily{song}\xiaowu \leftmark}%
  \fancyhead[RE]{\CJKfamily{song}\xiaowu  \cxuewei 论文}%
  \fancyfoot[RO,LE]{\xiaowu ~\thepage~} %页码左右两边有一小段横线%\if@mainmatter \fi
\else
\fancyhead{} % clear all fields
  \fancyhead[LO]{\CJKfamily{song}\xiaowu{}\cxuewei 论文}
  \fancyhead[RO]{\CJKfamily{song}\xiaowu{}基于图的模式识别及其在计算机视觉中的应用}
  \fancyhead[LE]{\CJKfamily{song}\xiaowu\leftmark }%
  \fancyhead[RE]{\CJKfamily{song}\xiaowu  \cxuewei 论文 }%
  \fancyfoot[RO,LE]{\xiaowu ~\thepage~} %\if@mainmatter \fi
\fi

\renewcommand\frontmatter{%
    \cleardoublepage
  \@mainmatterfalse
  \pagenumbering{Roman}}
%%%%%%%%%%%%%%%%%%%%%%%%%%%%%%%%%%%%%%%%%%%%%%%%%%%%%%%%
% 设置行距和段落间垂直距离
%%%%%%%%%%%%%%%%%%%%%%%%%%%%%%%%%%%%%%%%%%%%%%%%%%%%%%%%
\renewcommand{\CJKglue}{\hskip 0.3pt plus 0.08\baselineskip}%加大字间距,使每行33个字
%\setlength{\belowcaptionskip}{10pt}   % 加大标题和表格之间的距离 \abovecaptionskip 默认是10pt
\setlength{\parskip}{3pt plus1pt minus1pt} % 段落之间的竖直距离
\renewcommand{\baselinestretch}{2}% 定义行距
%%%%%%%%%%%%%%%%%%%%%%%%%%%%%%%%%%%%%%%%%%%%%%%%%%%%%%%%
% 调整列表环境的垂直间距
%%%%%%%%%%%%%%%%%%%%%%%%%%%%%%%%%%%%%%%%%%%%%%%%%%%%%%%%
\setitemize{itemindent=38pt,leftmargin=0pt,itemsep=-0.4ex,listparindent=26pt,partopsep=0pt,parsep=0.5ex,topsep=-0.25ex}
\setenumerate{itemindent=38pt,leftmargin=0pt,itemsep=-0.4ex,listparindent=26pt,partopsep=0pt,parsep=0.5ex,topsep=-0.25ex}
\setdescription{itemindent=38pt,leftmargin=0pt,itemsep=-0.4ex,listparindent=26pt,partopsep=0pt,parsep=0.5ex,topsep=-0.25ex}

\newcommand{\ucite}[1]{$^{\mbox{\scriptsize \cite{#1}}}$} % 增加 \ucite 命令使显示的引用为上标形式
\newcommand{\citeup}[1]{$^{\mbox{\scriptsize \cite{#1}}}$} % for WinEdt users

%%%%%%%%%%%%%%%%%%%%%%%%%%%%%%%%%%%%%%%%%%%%%%%%%%%%%%%%%%%
% 定制浮动图形和表格标题样式 %这里用ccaption完全代替了caption2的功能
\captionstyle{\centering}   %不同的图标题形式采用不同的命令
%\indentcaption{0pt}           %参见ccaption
\hangcaption
\captionnamefont{\CJKfamily{song}\rmfamily\wuhao\selectfont}
\captiontitlefont{\CJKfamily{song}\rmfamily\wuhao\selectfont}
\captiondelim{~} %~

%%%%%%%%%%%%%%%%%%%%%%%%%%%%%%%%%%%%%%%%%%%%%%%%%%%%%%%
% 定义题头格言的格式
% 用法 \begin{Aphorism}{author}
%         aphorism
%      \end{Aphorism}
\newsavebox{\AphorismAuthor}
\newenvironment{Aphorism}[1]
{\vspace{0.5cm}\begin{sloppypar} \slshape
\sbox{\AphorismAuthor}{#1}
\begin{quote}\small\itshape }
{\\ \hspace*{\fill}------\hspace{0.2cm} \usebox{\AphorismAuthor}
\end{quote}
\end{sloppypar}\vspace{0.5cm}}

%自定义一个空命令,用于注释掉文本中不需要的部分。
\newcommand{\comment}[1]{}

\renewcommand\contentsname{\hei 目~~~~录}
\renewcommand\listfigurename{\hei 插图目录}
\renewcommand\listtablename{\hei 表格目录}

%%%%%%将章标题中的中文数字(一、二、三)变为阿拉伯数字(1,2,3)
\renewcommand\CJKthechapter{%\CJKnumber
{\@arabic\c@chapter}}

%%%%%%不要拉大行距使得页面充满
\raggedbottom

% This is the flag for longer version
\newcommand{\longer}[2]{#1}

\newcommand{\ds}{\displaystyle}

% define graph scale
\def\gs{1.0}

%%%%%%%%%%%%%%%%%%%%%%%%%%%%%%%%%%%%%%%%%%%%%%%%%%%%%%%%%%%%%%%%%%%%%%
% 自定义项目列表标签及格式 \begin{hitlist} 列表项 \end{hitlist}
%%%%%%%%%%%%%%%%%%%%%%%%%%%%%%%%%%%%%%%%%%%%%%%%%%%%%%%%%%%%%%%%%%%%%%
\newcounter{hitctr} %自定义新计数器
\newenvironment{hitlist}{%%%%%定义新环境
\begin{list}{{\hei (\arabic{hitctr})}} %%标签格式
    {
     \usecounter{hitctr}
     \setlength{\leftmargin}{0cm}     %左边界
     \setlength{\parsep}{0ex}         %段落间距
     \setlength{\topsep}{0pt}         %列表到上下文的垂直距离
     \setlength{\itemsep}{0ex}        %标签间距
     \setlength{\labelsep}{0.3em}     %标号和列表项之间的距离,默认0.5em
     \setlength{\itemindent}{46pt}    %标签缩进量
     \setlength{\listparindent}{27pt} %段落缩进量
    }}
{\end{list}}%%%%%

%%%%%%%%%%%%%%%%%%%%%%%%%%%%%%%%%%%%%%%%%%%%%%%%%%%%%%%%%%%%%%%%%%%%%%
% 自定义项目列表标签及格式 \begin{publist} 列表项 \end{publist}
%%%%%%%%%%%%%%%%%%%%%%%%%%%%%%%%%%%%%%%%%%%%%%%%%%%%%%%%%%%%%%%%%%%%%%
\newcounter{pubctr} %自定义新计数器
\newenvironment{publist}{%%%%%定义新环境
\begin{list}{\arabic{pubctr}} %%标签格式
    {
     \usecounter{pubctr}
     \setlength{\leftmargin}{2em}     % 左边界 \leftmargin =\itemindent + \labelwidth + \labelsep
     \setlength{\itemindent}{0em}     % 标号缩进量
     \setlength{\labelwidth}{1em}     % 标号宽度
     \setlength{\labelsep}{1em}       % 标号和列表项之间的距离,默认0.5em
     \setlength{\rightmargin}{0em}    % 右边界
     \setlength{\topsep}{0ex}         % 列表到上下文的垂直距离
%     \setlength{\partopsep}{0ex}      % 列表是一个新的段落时,加的额外到上下文的距离
     \setlength{\parsep}{0ex}         % 段落间距
     \setlength{\itemsep}{0ex}        % 标签间距
     \setlength{\listparindent}{26pt} % 段落缩进量
    }}
{\end{list}}%%%%%

%%%%%%%%%%%%%%%%%%%%%%%%%%%%%%%%%%%%%%%%%%%%%%%%%%%%%%%%%%%%%%%%%%%%%%
% 默认字体
\renewcommand\normalsize{%
  \@setfontsize\normalsize{12.1pt}{13pt}
  \setlength\abovedisplayskip{8pt plus 2pt minus 2pt}
  \setlength\abovedisplayshortskip{7pt plus 2pt minus 2pt}
  \setlength\belowdisplayskip{\abovedisplayskip}
  \setlength\belowdisplayshortskip{\abovedisplayshortskip}
  \setlength\jot{6pt}
  \let\@listi\@listI}
\def\defaultfont{\renewcommand{\baselinestretch}{1.37}\normalsize\selectfont}
\predisplaypenalty=0  %公式之前可以换页,公式出现在页面顶部
%%%%%%%%%%%%%%%%%%%%%%%%%%%%%%%%%%%%%%%%%%%%%%%%%%%%%%%%%%
% 定义可以指定宽度的下划线
\def\NJUSTunderline[#1]#2{
\underline{\hbox to #1{\hfill#2\hfill}}}\def\@NJUSTunderline{}
%%%%%%%%%%%%%%%%%%%%%%%%%%%%%%%%%%%%%%%%%%%%%%%%%%%%%%%%%%%%%%%%%%%%%%
% 封面、摘要、版权、致谢格式定义
%--------定义标签对应选项的值-----------------------------
\def\natclassifiedindex#1{\def\@natclassifiedindex{#1}}\def\@natclassifiedindex{}
\def\secretclassifiedindex#1{\def\@secretclassifiedindex{#1}}\def\@secretclassifiedindex{}%密级
\def\internatclassifiedindex#1{\def\@internatclassifiedindex{#1}}\def\@internatclassifiedindex{}

\def\cname#1{\def\@cname{#1}}\def\@cname{}
\def\crole#1{\def\@crole{#1}}\def\@crole{}
\def\school#1{\def\@school{#1}}\def\@school{}
\def\ctitle#1{\def\@ctitle{#1}}\def\@ctitle{}
\def\cdegree#1{\def\@cdegree{#1}}\def\@cdegree{}
\def\csubject#1{\def\@csubject{#1}}\def\@csubject{}
\def\cauthor#1{\def\@cauthor{#1}}\def\@cauthor{}
\def\csupervisor#1{\def\@csupervisor{#1}}\def\@csupervisor{}
%\def\cassosupervisor#1{\def\@cassosupervisor{~ & {\hei 副 \hfill 导 \hfill 师:} & #1\\}}\def\@cassosupervisor{}
%\def\ccosupervisor#1{\def\@ccosupervisor{~ & {\hei 联 \hfill 合\hfill 导 \hfill 师:} & #1\\}}\def\@ccosupervisor{}
\def\cdate#1{\def\@cdate{#1}}\def\@cdate{}
\def\ddate#1{\def\@ddate{#1}}\def\@ddate{}%定义答辩日期
\def\chairman#1{\def\@chairman{#1}}\def\@chairman{}
\def\expositor#1{\def\@expositor{#1}}\def\@expositor{}


%--------------定义标签(label)---------------------
\def\lnatclassifiedindex#1{\def\@lnatclassifiedindex{#1}}\def\@lnatclassifiedindex{}
\def\lsecretclassifiedindex#1{\def\@lsecretclassifiedindex{#1}}\def\@lsecretclassifiedindex{}%密级
\def\linternatclassifiedindex#1{\def\@linternatclassifiedindex{#1}}\def\@linternatclassifiedindex{}

\def\lname#1{\def\@lname{#1}}\def\@lname{}
\def\lauthor#1{\def\@lauthor{#1}}\def\@lauthor{}
\def\lschool#1{\def\@lschool{#1}}\def\@lschool{}
\def\ldegree#1{\def\@ldegree{#1}}\def\@ldegree{}
\def\lsubject#1{\def\@lsubject{#1}}\def\@lsubject{}
\def\lsupervisor#1{\def\@lsupervisor{#1}}\def\@lsupervisor{}
%\def\lassosupervisor#1{\def\@lassosupervisor{~ & {\hei 副 \hfill 导 \hfill 师:} & #1\\}}\def\@lassosupervisor{}
%\def\lcosupervisor#1{\def\@lcosupervisor{~ & {\hei 联 \hfill 合\hfill 导 \hfill 师:} & #1\\}}\def\@lcosupervisor{}
\def\ldate#1{\def\@ldate{#1}}\def\@ldate{}
\def\lddate#1{\def\@lddate{#1}}\def\@lddate{}%定义答辩日期
\def\lchairman#1{\def\@lchairman{#1}}\def\@lchairman{}
\def\lexpositor#1{\def\@lexpositor{#1}}\def\@lexpositor{}

\long\def\cabstract#1{\long\def\@cabstract{#1}}\long\def\@cabstract{}%摘要标签
\def\ckeywords#1{\def\@ckeywords{#1}}\def\@ckeywords{}%关键词标签

%--------------定义英文封面---------------------
\def\etitle#1{\def\@etitle{#1}}\def\@etitle{}
\def\edegree#1{\def\@edegree{#1}}\def\@edegree{}
\def\eaffil#1{\def\@eaffil{#1}}\def\@eaffil{}
\def\esubject#1{\def\@esubject{#1}}\def\@esubject{}
\def\eauthor#1{\def\@eauthor{#1}}\def\@eauthor{}
\def\esupervisor#1{\def\@esupervisor{#1}}\def\@esupervisor{}
%\def\eassosupervisor#1{\def\@eassosupervisor{#1}}\def\@eassosupervisor{}
%\def\eassosupervisor#1{\def\@eassosupervisor{~ & \textbf{Associate Supervisor:} & #1\\}}\def\@eassosupervisor{}
\def\ecosupervisor#1{\def\@ecosupervisor{#1}}\def\@ecosupervisor{}
%\def\ecosupervisor#1{\def\@ecosupervisor{~ & \textbf{Co Supervisor:} & #1\\}}\def\@ecosupervisor{}
\def\edate#1{\def\@edate{#1}}\def\@edate{}
\long\def\eabstract#1{\long\def\@eabstract{#1}}\long\def\@eabstract{}
\long\def\NotationList#1{\long\def\@NotationList{#1}}\long\def\@NotationList{}
\def\ekeywords#1{\def\@ekeywords{#1}}\def\@ekeywords{}
%%%%%%%%%%%%%%%%%%%%%%%%%%%%%%%%%%%%%%%%%%%%%%%%%%%%%%%%%%%%%%%
% 定义封面
\def\makecover{
    \normalbiao %表格字号设置
     %封面一
    \newpage
    \thispagestyle{empty}
    \begin{center}

      \parbox[t][0.6cm][t]{\textwidth}{
      \begin{center} \end{center}}

      \parbox[t][2.2cm][t]{\textwidth}{%
       \song \wuhao
       \@lnatclassifiedindex~
       \NJUSTunderline[100pt]{\@natclassifiedindex }
        \hfill
       \@lsecretclassifiedindex~
       \NJUSTunderline[100pt]{\@secretclassifiedindex}
        \vskip 10pt
        \@linternatclassifiedindex~
        \NJUSTunderline[108pt]{\@internatclassifiedindex}\hfill
       }%

      \parbox[t][2.7cm][b]{\textwidth}{\song\yihao
      \begin{center} {\textbf{学\qquad 位\qquad 论\qquad 文}}\end{center} }

      \setlength{\baselineskip}{1.5\baselineskip}
      \parbox[t][3.0cm][b]{\textwidth}{\erhao
      \begin{center} {\hei\textbf{\@school\@cdegree\@ctitle}}\end{center} }

      \begin{center} {\xiaoer\kai{\NJUSTunderline[100pt]{\textbf{\@cauthor}}}} \end{center}%还可设置上下间隔

      \parbox[t][0.8cm][t]{\textwidth}{\begin{center}  \end{center} }

      \parbox[t][6cm][c]{\textwidth}{
      \begin{center}
        \begin{tabular}{llcll}
        \song\xiaosi\@lsupervisor\@lname~&
        \multicolumn{4}{l}{\NJUSTunderline[320pt]{\kai\sanhao\textbf{\@csupervisor}\hspace*{2em}\kai\sihao\@crole}}\vspace{18pt}\\
        \hspace*{2.5cm}&
        \multicolumn{4}{l}{\NJUSTunderline[320pt]{}}\vspace{18pt}\\
        %\NJUSTunderline[115pt]{\@ccosupervisor}
        %\NJUSTunderline[115pt]{\@cassosupervisor}
        \song\xiaosi\@ldegree &
        \NJUSTunderline[100pt]{\kai\sanhao\textbf{\@cdegree}}&\hspace*{1pt}&
        \song\xiaosi\@lsubject~&
        \NJUSTunderline[100pt]{\kai\sanhao\textbf{\@csubject}}\vspace{18pt}\\
        \song\xiaosi\@ldate~&
        \NJUSTunderline[100pt]{\defaultfont\@cdate}&\hspace*{2pt}&
        \song\xiaosi\@lddate~&
        \NJUSTunderline[100pt]{\defaultfont\@ddate}\vspace{18pt}\\
        \multicolumn{5}{l}{\song\xiaosi\@lschool~~\NJUSTunderline[285pt]{\kai\sanhao\@school}}\vspace{18pt}\\
        \multicolumn{4}{r}{\song\xiaosi\@lchairman}&
        \NJUSTunderline[100pt]{\@chairman}\vspace{18pt}\\
        \multicolumn{4}{r}{\song\xiaosi\@lexpositor}&
        \NJUSTunderline[100pt]{\@expositor}\vspace{18pt}
       \end{tabular}
     \end{center}}

   \end{center}
%%%%%%封面一增加一空白页
  \ifoneortwoside
    \newpage
    ~~~\vspace{1em}
    \thispagestyle{empty}
  \fi
%%%%%%%%%内封
    \begin{titlepage}
    \begin{center}
      \parbox[t][1cm][b]{\textwidth}{\xiaoer
      \begin{center} {\kai \ifxueweimaster\cxueke\fi 中华人民共和国\cxueke\cxuewei 学位论文 }\end{center} }

      \parbox[t][0.8cm][t]{\textwidth}{
      \begin{center} \end{center} }

      \parbox[t][2cm][t]{\textwidth}{\xiaoer
      \begin{center} {\hei\textbf{\@school\@cdegree\@ctitle}}\end{center} }

      \ifxueweidoctor
      \parbox[t][3.8cm][t]{\textwidth}{\xiaoer
      \begin{center} {\hei  \@etitle}\end{center} }
      \else
      \parbox[t][3.8cm][t]{\textwidth}{\centering
        \ }
      \fi

    \parbox[t][2.0cm][t]{\textwidth}{\kai\xiaoer
    \begin{center}
    \begin{tabular}{lcc}
    \textbf{\@lauthor:}&
    \multicolumn{2}{l}{\textbf{\@cauthor}}\\
    \textbf{\@lsupervisor:}&
    \textbf{\@csupervisor }&
    \textbf{\@crole}
    \end{tabular}
    \end{center}}

    \ifxueweidoctor
    %\parbox[t][8.5cm][t]{\textwidth}{\centering
    \parbox[c][8.5cm][c]{\textwidth}{\centering
        \includegraphics[width = 7cm,height=7cm]{xiaohui}}
    \else
    \parbox[t][8.5cm][t]{\textwidth}{\centering \ }
    \fi

    \parbox[t][1.2cm][t]{\textwidth}{\xiaoer
    \begin{center} {\song  \textbf{\@school}}  \end{center} }

    \parbox[t][0.5cm][t]{\textwidth}{
    \begin{center} {\song \xiaoer \textbf{\@cdate}} \end{center} }
   \end{center}

%%%% 内封增加一空白页
    \ifoneortwoside
      \newpage
      ~~~\vspace{1em}
      \thispagestyle{empty}
    \fi

%%%%% 英文封面
    \newpage
    \thispagestyle{empty}
    \begin{center}
    \parbox[t][0.6cm][t]{\textwidth}{
    \begin{center} \end{center}}

   \parbox[t][2.2cm][t]{\textwidth}{}

    \parbox[t][1cm][b]{\textwidth}{\xiaoer
    \begin{center} {PH. D. Dissertation}\end{center} }

    \parbox[t][2mm][t]{\textwidth}{
    \begin{center}  \end{center}}

    \parbox[t][3.0cm][b]{\textwidth}{\erhao
    \begin{center} {\textbf{\@etitle}}\end{center} }

    \parbox[t][3.0cm][t]{\textwidth}{
    \begin{center}  \end{center}}

    \parbox[t][2.5mm][b]{\textwidth}{\xiaoer
    \begin{center} { \emph{By}}\end{center}

    \begin{center} { \emph{\textbf{\@eauthor}}}\end{center}}

    \parbox[t][1.2cm][b]{\textwidth}{\xiaoer
    \begin{center} {\emph{Supervised by Prof.~\textbf{\@esupervisor}}}\end{center}}

    \parbox[t][6.0cm][t]{\textwidth}{
    \begin{center}  \end{center} }

    \parbox[t][2.6cm][b]{\textwidth}{\xiaoer
    \begin{center} {Nanjing University of  Science \& Technology}\end{center}

    \begin{center} {\@edate }\end{center}}

   \end{center}
  \end{titlepage}
%%%%%%英文封面增加一空白页
  \ifoneortwoside
    \newpage
    ~~~\vspace{1em}
    \thispagestyle{empty}
  \fi

 \thispagestyle{empty}
 % !Mode:: "TeX:UTF-8"
% 这个是为了WinEdt设置的,它的默认不是UTF8.
% !TeX root = ../main.tex
% 若xelatex编译非UTF8文件,需在每个文件中指定字符编码;
% main.tex中手动制定了\atemp和\usewhat参数;
\ifx\atempxetex\usewhat
%\XeTeXinputencoding "gbk"
\fi
\defaultfont

\thispagestyle{empty}
%在此节中,由于用了与其它正文不同的字体,此节中用的是
%四号字体,所以要对行距与首行缩进重新定义。
\begin{center}{\sanhao \hei{声\quad 明}}\end{center}

\renewcommand{\baselinestretch}{1.5}\large{}
{\setlength{\parindent}{2em}本学位论文是我在导师的指导下取得的研究成果,尽我所知,在本
 学位论文中,除了加以标注和致谢的部分外,不包含其他人已经发表或
公布过的研究成果,也不包含我为获得任何教育机构的学位或学历而使
用过的材料。与我一同工作的同事对本学位论文做出的贡献均已在论文
中作了明确的说明。}

    \vspace{0.738cm}
    \begin{flushleft}{
    研究生签名:\underline{~~~~~~~~~~~~~~~}~~~~~~~~~~~~~~~~日期:~~~~~~~~~~~年~~~~~月~~~~~日}
    \end{flushleft}

    \vspace{2.214cm}
%%%%%%%%%%%%%%%%%%南京理工大学博(硕)士学位论文使用授权书%%%%%%%%%%%%%%%%%%%
%\phantomsection
    \begin{center}{\sanhao \hei{南京理工大学\cxuewei 学位论文使用授权声明}}
    \end{center}

    \vspace{0.738cm}

\renewcommand{\baselinestretch}{1.5}\large{}
{\setlength{\parindent}{2em}南京理工大学有权保存本学位论文的电子和纸质文档,可以借阅或上网公布本学位论文的部分或全部内容,
可以向有关部门或机构送交并授权其保存、借阅或上网公布本学位论文的部分或全部内容。对于保密论文,
按保密的有关规定和程序处理。}


\vspace{1.476cm}
\begin{flushleft}{
研究生签名:\underline{~~~~~~~~~~~~~~~~~}~~~~~~~~~~~~~~日期:~~~~~~~~~~~年~~~~~月~~~~~日}
\end{flushleft}
\vspace{0.2cm}
\iffalse
\begin{flushleft}{
\hspace*{8pt}导师签名:\underline{~~~~~~~~~~~~~~~~~~~}~~~~~~~~~~~~~~日期:~~~~~~~~~~~年~~~~~月~~~~~日}
\end{flushleft}
\fi
\newpage
   % 原创性申明,即让原创性申明在摘要前显示

%%%%%%增加一空白页
 \ifoneortwoside
    \newpage
    ~~~\vspace{1em}
    \thispagestyle{empty}
  \fi
%%%%%%%%%%%%%%%%%%%   Abstract and keywords  %%%%%%%%%%%%%%%%%%%%%%%修改摘要和关键词???????????
\clearpage \BiAppendixChapter{\textbf{摘\hspace*{2em}要}}{Abstract (in Chinese)} %不要挪到下一行,生成正确的摘要toc
\setcounter{page}{1}
\song \normalsize
\defaultfont
\@cabstract \vspace{1em} \hangafter1\hangindent4.28em\noindent
{\sihao\song \textbf{关键词}} \quad \@ckeywords

%%%%%%%%%%%%%%%%%%%   English Abstract  %%%%%%%%%%%%%%%%%%%%%%%%%%%%%%
\clearpage
\defaultfont \BiAppendixChapter{\textbf{Abstract}}{Abstract (in English)} %不要挪到下一行,生成正确的摘要toc
\@eabstract
\vspace{1em}

\hangafter1\hangindent5.5em\noindent
{\textbf{Keywords}} \quad \@ekeywords
\wuhaobiao  %正文表格设置
}

%%%%%%%%%%%%%%%%%%%%%%%%%%%%%%%%%%%%%%%%%%%%%%%%%%%%%%%%%%%%%%%

\urlstyle{same}  %论文中引用的网址的字体默认与正文中字体不一致,这里修正为一致的。

%主要符号表 \NotationList
\long\def\notation{ \clearpage
\BiAppendixChapter{\textbf{主要符号说明}}{} \normalbiao
\@NotationList \wuhaobiao}

%%% 五号字表格设置 start %%%%%
\gdef\tpltable{\relax}
\let\tpltable\longtable
\gdef\wuhaobiao{%五号字
    \def\tabular{\wuhao\gdef\@halignto{}\@tabular}
    \def\endtabular{\endarray $\egroup \defaultfont}
    \def\longtable{\wuhao\tpltable}
    \def\endlongtable{\adl@LTlastrow \adl@org@endlongtable\defaultfont}
}
\gdef\normalbiao{%正常字号
    \def\tabular{\gdef\@halignto{}\@tabular}
    \def\endtabular{\endarray $\egroup}
    \def\longtable{\tpltable}
    \def\endlongtable{\adl@LTlastrow \adl@org@endlongtable}
}
\wuhaobiao
%%% 五号字表格设置 end %%%%%
%\renewcommand{\arraystretch}{1.4} %表格中行距 ,导致公式 bmatrix 间距增大。

% 表格与下方间距
\renewcommand\endtable{\vspace{-2mm}\end@float}
% 算法与下方间距
\renewcommand\endalgorithm{\@algocf@finish \ifthenelse {\equal {\algocf@float }{figure}}{\end {figure}}{
\@algocf@term@caption \ifthenelse {\boolean {algocf@algoH}}{\end {algocf@Here}}
{\end {algocf}}}\@algocf@term\vspace{-5mm}}

\makeatother


%%%%%%%%%%%%%%%%%%%%%%%%%%%%%%%%%%%%%%%%%%%%%%%%%%
% 正文部分
%%%%%%%%%%%%%%%%%%%%%%%%%%%%%%%%%%%%%%%%%%%%%%%%%%
\frontmatter
\sloppy % 解决中英文混排的断行问题,会加入间距,但不会影响断行
%\interlinepenalty -100000
% !Mode:: "TeX:UTF-8"
% 这个是为了WinEdt设置的,它的默认不是UTF8.
% !TeX root = ../main.tex
% 这个为TexWorks设置,可以用来编译多个文件。

% 若xelatex编译非UTF8文件,需在每个文件中指定字符编码;
% main.tex中手动制定了\atemp和\usewhat参数;
\ifx\atempxetex\usewhat
%\XeTeXinputencoding "gbk"
\fi

\newcommand{\chinesethesistitle}{南京理工大学博士学位论文模板} %授权书用,无需断行
\newcommand{\englishthesistitle}{Phd Thesis LaTeX Template for \\Nanjing Unversity of Science and Technology}
\newcommand{\chinesethesistime}{2011~年~10~月}  %封面底部的日期中文形式
\newcommand{\dchinesethesistime}{2011~年~12~月} %答辩日期中文形式
\newcommand{\englishthesistime}{October, 2011}   %封面底部的日期英文形式
%-----定义标签值-与format.tex文件中的定义相对应----------------
\lnatclassifiedindex{分类号}
\lsecretclassifiedindex{密级}
\linternatclassifiedindex{UDC}

\lsupervisor{指导教师}
\lname{姓名}
\lauthor{作\hspace*{2em}者}
\ldegree{申请学位级别}
\lsubject{专业名称}
\ldate{论文提交日期}
\lddate{论文答辩日期}
\lschool{学位授予单位和日期}
\lchairman{答辩委员会主席}
\lexpositor{评阅人}
\cabstract{摘\hspace*{2em}要}
\ckeywords{关键词}

%----定义标签对应选项的值---------------
\natclassifiedindex{TP309}
\secretclassifiedindex{公开}
\internatclassifiedindex{~681.324} %国际图书分类号

%\school{南京理工大学}
%\cdegree{\cxueke\cxuewei}
\ctitle{\erhao\hei 南京理工大学博士学位论文模板}  %封面用论文标题,自己可手动断行
\csubject{专业名称}                 %(~按二级学科填写~)
\cauthor{某学生}
\crole{教授}
\csupervisor{某~教~授} %导师名字
%\cassosupervisor{某~~~~~~某~~~~教~~授}     %(~如无副导师可以不列此项~)
%\ccosupervisor{某~~某~~某~~~~教~~授~} %(~如无联合培养导师则不列此项~)
\cdate{\chinesethesistime}
\ddate{\dchinesethesistime}%定义答辩日期

\etitle{\englishthesistitle}
\edegree{\exuewei \ of \exueke}
\esubject{Zhuanye mingcheng}  %英文二级学科名
\eaffil{School of Science and Technology}%英文单位 %换行用\newline,不要用\\
\eauthor{\textbf{Student Name}}                   %作者姓名 (英文)
\esupervisor{\textbf{Supervisor's Name}}       % 导师姓名 (英文)
%\ecosupervisor{Professor X}
%\eassosupervisor{Professor Y}
\edate{\englishthesistime}


\cabstract{

背景

主要贡献


%关键词不能放在摘要页的外面
 \ckeywords{LaTeX \quad 南京理工大学 \quad 博士论文模板 }
}

\eabstract{ 

English Abstract
}

\ekeywords{LaTeX \quad Nanjing University of Scicence and Technology \quad Phd Thesis Template}

%主要符号说明
%\NotationList{
%%\vspace*{10pt}%可在主要符号说明与下面内容之间留空
%\begin{center}
%\begin{tabular}{llll}
%面积,& $m^2$             & 希腊字母  &\\
%\multicolumn{1}{c}{$c_p$}  & 定压比热,~$J/(kg \cdot{k})$~& \multicolumn{1}{c}{$\mu{}$}~& 动力粘度,~$kg/(m\cdot{s})$~\\
%\multicolumn{1}{c}{$\emph{d}$} & 直径,~$m$~ & \multicolumn{1}{c}{$\alpha{}$} & 总热适应系数~\\
%~& 全导数 &  \multicolumn{1}{c}{$\omega{}$} & 孔隙率~\\
%\multicolumn{1}{c}{$\emph{f}$} & 摩擦因子 & \multicolumn{1}{c}{$\varepsilon$} & 相含量~\\
%\multicolumn{1}{c}{$\emph{h}$} & 对流换热系数,~$w/{m^2}\cdot k$~ & & 发射率~\\
%~& 比焓,~${J/kg}$~ & \multicolumn{1}{c}{$\delta$} & 系数变量~\\
%\multicolumn{1}{c}{$\emph{H}$} & 焓,~$J$~ &\multicolumn{1}{c}{$\nu$}& 运动粘度
%\end{tabular}
%\end{center}
%}
\makecover
\clearpage
 % 封面

%% 中文目录
\renewcommand{\baselinestretch}{1}
\fontsize{12pt}{12pt}\selectfont
\clearpage{\pagestyle{empty}\cleardoublepage}
\pdfbookmark[0]{目~~~~录}{mulu}
\tableofcontents    % 中文目录

% !Mode:: "TeX:UTF-8"
% !TeX root = ../main.tex
% 这个是为了WinEdt设置的,它的默认不是UTF8.
% 若xelatex编译非UTF8文件,需在每个文件中指定字符编码;
% main.tex中手动制定了\atemp和\usewhat参数;
\ifx\atempxetex\usewhat
%\XeTeXinputencoding "gbk"
\fi

%% 中英文图形表格索引   %硕博士学位论文规范均不要求这一项,请正式打印的时候屏蔽掉这一段;
\ifxueweidoctor
  %\clearpage{\pagestyle{empty}\cleardoublepage}   % 清除目录后面空页的页眉和页脚
\else%
  {\ifoneortwoside\clearpage{\pagestyle{empty}\cleardoublepage}\else\newpage\fi} % 清除目录后面空页的页眉和页脚
\fi
\addcontentsline{toc}{chapter}{\textbf{插图目录}}   % 中文插图加入到中文目录
\listoffigures                                  % 生成中文 图形索引
\ifxueweidoctor                                 %硕士学位论文没有英文目录
%\clearpage{\pagestyle{empty}\cleardoublepage}   % 英文图形索引 右开 ?需要吗?
\fi

\ifxueweidoctor
  %\clearpage{\pagestyle{empty}\cleardoublepage}   % 清除目录后面空页的页眉和页脚
\else%
  {\ifoneortwoside\clearpage{\pagestyle{empty}\cleardoublepage}\else\newpage\fi}   % 清除目录后面空页的页眉和页脚
\fi
\addcontentsline{toc}{chapter}{\textbf{表格目录}}   % 中文表格加入到中文目录
\listoftables                                   % 生成中文 表格索引
\ifxueweidoctor                                 %硕士学位论文没有英文目录
%\clearpage{\pagestyle{empty}\cleardoublepage}   % 英文表格索引 右开 ?需要吗?
\fi
%% 如果不需要图表索引,注释掉上面的即可
  %图表索引, 如果不需要图表索引,注释掉这一句即可;
%\notation  %主要符号表
\addtocontents{toc}{\protect\vskip1\baselineskip} % 中文目录增加空行

\ifxueweidoctor
  \clearpage{\pagestyle{empty}\cleardoublepage}   % 清除目录后面空页的页眉和页脚
\else%
  \ifoneortwoside\clearpage{\pagestyle{empty}\cleardoublepage}\fi  % 清除目录后面空页的页眉和页脚
\fi                                               %  第一章 是否右开

\mainmatter
\defaultfont % 对应于小四的标准字号是 12pt, 可以在正文中用此命令修改所需要字体的的大小

% !Mode:: "TeX:UTF-8"
% 这个是为了WinEdt设置的,它的默认不是UTF8.
% !TeX root = ../main.tex
% 这个为TexWorks设置,可以用来编译多个文件。

% 若xelatex编译非UTF8文件,需在每个文件中指定字符编码;
% main.tex中手动制定了\atemp和\usewhat参数;
\ifx\atempxetex\usewhat
%\XeTeXinputencoding "gbk"
\fi

\defaultfont

\titleformat{\chapter}[hang]{\xiaosan\bf\raggedright\song\sf\boldmath}{\xiaoer\chaptertitlename}{18pt}{\xiaosan}
\titlespacing{\chapter}{0pt}{8pt}{16pt}

\makeatletter
\newskip\@footindent
\@footindent=1em

\renewcommand\footnoterule{\kern-3\p@ \hrule width 0.4\columnwidth \kern 2.6\p@}
\@addtoreset{footnote}{page}

\long\def\@makefntext#1{\@setpar{\@@par\@tempdima \hsize
\advance\@tempdima-\@footindent
\parshape \@ne \@footindent \@tempdima}\par
\noindent \hbox to \z@{\hss\@thefnmark\hspace{0.5em}}#1}

\renewcommand\thefootnote{\pinumber{\arabic{footnote}}}
\def\@makefnmark{\hbox{\textsuperscript{\@thefnmark}}}

\newcommand\pinumber[1]{\ifcase#1 \or \ding{172}\or \ding{173}\or
  \ding{174}\or \ding{175}\or \ding{176}\or \ding{177}%
  \or \ding{178}\or \ding{179}\or \ding{180}\or \ding{181}\else *\fi\relax}
\makeatother
%以上从\makeatletter到\makeatother为重定义脚注编号,使之带圆圈

\chapter{绪论}
\label{cha1:introduction}


\section{研究背景}
\label{sec1:background}

\subsection{图论的兴起}
\label{subsec1:graph_theory}



随着科技的迅猛发展,越来越多的问题可以转化成为图,基于图的研究越来越多。
\cite{Lenard2011}




\subsection{当前数据的复杂性}
\label{subsec1:data_complexity}

\section{研究意义}
\label{sec1:motivation}


\section{本文的主要贡献}
\label{sec1:contribution}


\section{本文的组织结构}
\label{sec1:organization}

本文的主要结构如下。

第一章绪论,简要介绍本文研究工作的背景及意义,以及研究工作的意义,以及本文的主要贡献。

第二章文献综述。

最后结论,总结全文的主要工作,以及对于未来工作的展望。


% !Mode:: "TeX:UTF-8"
% !TeX root = ../main.tex
% 这个是为了WinEdt设置的,它的默认不是UTF8.

% 若xelatex编译非UTF8文件,需在每个文件中指定字符编码;
% main.tex中手动制定了\atemp和\usewhat参数;
\ifx\atempxetex\usewhat
%\XeTeXinputencoding "gbk"
\fi
\defaultfont

\chapter{文献综述}
\label{cha2:literature_review}


\section{基本概念}
\label{sec2:basic_concepts}


\section{本章小结}
\label{sec2:conclusion}


% !Mode:: "TeX:UTF-8"
% !TeX root = ../main.tex
% 若xelatex编译非UTF8文件,需在每个文件中指定字符编码;
% main.tex中手动制定了\atemp和\usewhat参数;
\ifx\atempxetex\usewhat 
%\XeTeXinputencoding "gbk"
\fi
\defaultfont

%%%%%%%%%%%%%%%%%%%%%%%%%%%%%%%%%%%%%%%%%%%%%
\chapter{我的研究内容}
\label{cha:research_contents}

\section{引言}
\label{sec5:introduction}



\section{核心算法}
\label{sec5:algorithm}



\section{讨论和实现细节}
\label{sec5:discussion_implementation_issues}



\section{实验}
\label{sec:experiments}



\subsection{实验1}
\label{subsec5:exp1}




\subsection{实验2}
\label{sec:exp2}



\section{结论}
\label{sec5:conclusion}




\include{body/chapter4}
\include{body/chapter5}
% !Mode:: "TeX:UTF-8"
% !TeX root = ../main.tex
% 若xelatex编译非UTF8文件,需在每个文件中指定字符编码;
% main.tex中手动制定了\atemp和\usewhat参数;
\ifx\atempxetex\usewhat
%\XeTeXinputencoding "gbk"
\fi
\defaultfont


\titleformat{\chapter}[hang]{\xiaosan\bf\filcenter\hei\sf\boldmath}{\xiaoer\chaptertitlename}{18pt}{\xiaosan}
\titlespacing{\chapter}{0pt}{8pt}{16pt}

\BiAppendixChapter{结~~~~论}{Conclusion}

本文的结论。
   % 结论
% 若xelatex编译非UTF8文件,需在每个文件中指定字符编码;
% main.tex中手动制定了\atemp和\usewhat参数;
% !TeX root = ../main.tex
\ifx\atempxetex\usewhat
%\XeTeXinputencoding "gbk"
\fi
\defaultfont

\BiAppendixChapter{致~~~~谢}{Acknowledgement}

论文终于写完了,我的博士生涯也要结束了。这本博士论文,不仅仅是我个人几年来
研究的总结,更多地,这是融入了许多人的心血的结果。

%值此论文完成之际,谨向给予我无私帮助的老师和同学们致以诚挚的谢意。



%该论文模板是在~UFO@bbs.hit.edu.cn的《哈尔滨工业大学大学博士(硕士)论文模板》的基础上,
%并在很多人的帮助下完成的,在此一并向他们表示感谢。
%
%值此论文完成之际,谨向给予我无私帮助的老师和同学们致以诚挚的谢意!
%
%首先感谢我的导师{\bf 某某某}教授,本论文的研究工作正是在{\bf
%某}老师最初的建议下展开的。
%他在学术上不断进取、对人生理想执着追求的精神是我学习的榜样。{\bf
%某}老师对问题深刻的认识和深入浅出的讲解给我留下深刻印象。
%
%
%感谢{\bf 某某某}教授和{\bf 某某某}教授对我学习和工作的帮助,
%他们勤奋的工作作风、达观的人生态度都深深地感染了我。感谢{\bf
%某某某}教授和{\bf 某某某}教授对我学业和生活上的关心。和im.
%
%
%感谢博士生{\bf 某某某}、{\bf 某某某}、{\bf 某某某}、{\bf 某某某},
%给我的无私帮助和积极支持。感谢实验室所有的兄弟姐妹们,
%陪伴我度过了这长久的学习、研究阶段,帮助我解决问题,开拓思想。
%
%最后,特别要感谢我的亲人们,他们对我要求甚少,但给予我的都是关怀、支持和理解。
% 致谢

%参考文献
\defaultfont
\ifx\atempxetex\usewhat
\bibliographystyle{chinesebst2005}
\else
\bibliographystyle{chinesebst}
\fi
\addcontentsline{toc}{chapter}{\hei \ReferenceCName}      % 参考文献加入到中文目录
\addcontentsline{toe}{chapter}{\bfseries \ReferenceEName} % 参考文献加入到英文目录
\addtolength{\bibsep}{-0.8 em} %\nocite{*}
\bibliography{reference/reference}

%\addtocontents{fen}{\protect\vskip1\baselineskip}
%\addtocontents{ten}{\protect\vskip1\baselineskip}
%英文图形和表格索引里加入空白行,通常放在 % !Mode:: "TeX:UTF-8"
% 这个是为了WinEdt设置的,它的默认不是UTF8.
% !TeX root = ../main.tex
% 若xelatex编译非UTF8文件,需在每个文件中指定字符编码;
% main.tex中手动制定了\atemp和\usewhat参数;
\ifx\atempxetex\usewhat
%\XeTeXinputencoding "gbk"
\fi

% \defaultfont
% \appendix
% 
% %%%%%%%%%%%%%%%%%%%%%%%%%%%%%%%%%%%%%%%%%%%%%%%%%%%%%%%%%
% \BiAppChapter{带章节的附录}{Full Appendix}%
% 完整的附录内容,包含章节,公式,图表等
% 
% %%%%%%%%%%%%%%%%%%%%%%%%%%%%%%%%%%%%%%%%%%%%%%%%%%%%%%%%%
% \BiSection{附录节的内容}{Section in Appendix}
% 这是附录的节的内容
% 
% 附录中图的示例:
% \begin{figure}[h]
% \centering
% \includegraphics[width = 0.8\textwidth]{golfer}
% \FigureBiCaption{实验装置图~H}{Experiment setup H}
% \label{Figure:Appendix:Example1}
% \end{figure}
% 
% 附录中公式的示例:
% \begin{align}
% a & = b \times c \\
% E & = m c^2
% \end{align}
% 
% %\BiAppChapter{附录二}{appendix 2}
% %\BiAppChapter{附录三}{appendix 3}
% 附录A之前。
%区分开正文和附录的图形和表格,一般没有这个必要。

% !Mode:: "TeX:UTF-8"
% 这个是为了WinEdt设置的,它的默认不是UTF8.
% !TeX root = ../main.tex
% 若xelatex编译非UTF8文件,需在每个文件中指定字符编码;
% main.tex中手动制定了\atemp和\usewhat参数;
\ifx\atempxetex\usewhat
%\XeTeXinputencoding "gbk"
\fi

% \defaultfont
% \appendix
% 
% %%%%%%%%%%%%%%%%%%%%%%%%%%%%%%%%%%%%%%%%%%%%%%%%%%%%%%%%%
% \BiAppChapter{带章节的附录}{Full Appendix}%
% 完整的附录内容,包含章节,公式,图表等
% 
% %%%%%%%%%%%%%%%%%%%%%%%%%%%%%%%%%%%%%%%%%%%%%%%%%%%%%%%%%
% \BiSection{附录节的内容}{Section in Appendix}
% 这是附录的节的内容
% 
% 附录中图的示例:
% \begin{figure}[h]
% \centering
% \includegraphics[width = 0.8\textwidth]{golfer}
% \FigureBiCaption{实验装置图~H}{Experiment setup H}
% \label{Figure:Appendix:Example1}
% \end{figure}
% 
% 附录中公式的示例:
% \begin{align}
% a & = b \times c \\
% E & = m c^2
% \end{align}
% 
% %\BiAppChapter{附录二}{appendix 2}
% %\BiAppChapter{附录三}{appendix 3}
            % 附录A
% !Mode:: "TeX:UTF-8"
% 这个是为了WinEdt设置的,它的默认不是UTF8.
% !TeX root = ../main.tex
% 若xelatex编译非UTF8文件,需在每个文件中指定字符编码;
% main.tex中手动制定了\atemp和\usewhat参数;
\ifx\atempxetex\usewhat 
%\XeTeXinputencoding "gbk"
\fi
\defaultfont

\BiAppendixChapter{附~~~~录}{Appendix}

\noindent \large \textbf{作者在攻读博士学位期间发表的学术论文:}
\vspace{15pt}

\begin{enumerate}
% for the final version

\item Author1, Author2, Author3. “Paper Title”, Accepted by the Conference Name, 2011. (EI收录)

\item Author1, Author2, Author3. “Paper Title”, Journal Title Vol. xx, No. xx, 2011. (SCI、EI收录,影响因子:xxx,SCI检索号:xxx,EI检索号:xxx)

\item Author1, Author2, Author3. “Paper Title”, Journal Title Vol. xx, No. xx, 2011. (SCI、EI收录,影响因子:xxx,SCI检索号:xxx,EI检索号:xxx)

\item Author1, Author2, Author3. “Paper Title”, Journal Title Vol. xx, No. xx, 2011. (SCI、EI收录,影响因子:xxx,SCI检索号:xxx,EI检索号:xxx)

\end{enumerate}

\vspace{100pt}
\vspace{15pt}
\noindent \large \textbf{作者在攻读博士学位期间参与的主要科研项目:}
\vspace{15pt}

\begin{enumerate}
 \item 国家自然科学基金重点项目(xxx)“项目名称”,参与,结题。
 \item 国家自然科学基金面上项目(xxx)“项目名称”,主要研究人员,结题。
\end{enumerate}
    % 所发文章
% !Mode:: "TeX:UTF-8"
% 这个是为了WinEdt设置的,它的默认不是UTF8.
% !TeX root = ../main.tex
% 若xelatex编译非UTF8文件,需在每个文件中指定字符编码;
% main.tex中手动制定了\atemp和\usewhat参数;
\ifx\atempxetex\usewhat
%\XeTeXinputencoding "gbk"
\fi
% \defaultfont
% 
% \BiAppendixChapter{}{}
% 
% {\hei 科研项目}
% \begin{publist}


% \end{publist}
% 
% %{\hei 科研工作}
% %\begin{publist}
% %\item  2007~年~x~月--xxxx~年~x~月 ~~~ xxxx项目~~~~(编号xxx-xxx-xxx) 
% %\item  xxxx~年~x~月--xxxx~年~x~月 ~~~ xxxx项目~~~~(编号xxx-xxx-xxx)
% %\item  xxxx~年~x~月--xxxx~年~x~月 ~~~ xxxx项目~~~~(编号xxx-xxx-xxx)
% %\item  xxxx~年~x~月--xxxx~年~x~月 ~~~ xxxx项目~~~~(编号xxx-xxx-xxx)
% %\end{publist}
% %
% %{\hei 学术论文}
% %\begin{publist}
% %\item 在~xxxxxxx~等刊物发表论文多篇
% %\item 在~xxxxxxxxxxxxxxxx~等多个国际会议上发表论文多篇
% %\end{publist}
% 
          % 个人简历

\clearpage
\ifx\atempxetex\usewhat\else
\end{CJK*}
\fi

\end{document}
