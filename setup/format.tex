% !Mode:: "TeX:UTF-8"
% !TeX root = ../main.tex
% 这个是为了WinEdt设置的,它的默认不是UTF8.
% 若xelatex编译非UTF8文件,需在每个文件中指定字符编码;
% main.tex中手动制定了\atemp和\usewhat参数;
\ifx\atempxetex\usewhat
%\XeTeXinputencoding "gbk"
\fi

%%%%%%%%%%%%%%%%%%%%%%%%%%%%%%%%%%%%%%%%%%%%%%%%%%%%%%%%%%%
%允许公式换页显示,否则大型推导公式都在一页内,
%一页显示不下放到第二页,导致很大的空白空间,很不好看
\allowdisplaybreaks[4]

%%%%%%%%%%%%%%%%%%%%%%%%%%%%%%%%%%%%%%%%%%%%%%%%%%%%%%%%%%%
%下面这组命令使浮动对象的缺省值稍微宽松一点,从而防止幅度
%对象占据过多的文本页面,也可以防止在很大空白的浮动页上放置很小的图形。
\renewcommand{\topfraction}{0.9999999}
\renewcommand{\textfraction}{0.0000001}
\renewcommand{\floatpagefraction}{0.9999}

%%%%%%%%%%%%%%%%%%%%%%%%%%%%%%%%%%%%%%%%%%%%%%%%%%%%%%%%%%%
% 重定义一些正文相关标题
%%%%%%%%%%%%%%%%%%%%%%%%%%%%%%%%%%%%%%%%%%%%%%%%%%%%%%%%%%%
\theoremstyle{plain} \theorembodyfont{\song\rmfamily}
\theoremheaderfont{\hei\rmfamily} %\theoremseparator{:}
\newtheorem{definition}{\hei 定义}[chapter]
\newtheorem{example}{\hei 例}[chapter]
\newtheorem{algo}{\hei 算法}[chapter]
\newtheorem{theorem}{\hei 定理}[chapter]
\newtheorem{axiom}{\hei 公理}[chapter]
\newtheorem{proposition}{\hei 命题}[chapter]
\newtheorem{lemma}{\hei 引理}[chapter]
\newtheorem{corollary}{\hei 推论}[chapter]
\newtheorem{remark}{\hei 注解}[chapter]
%\newtheorem{proposition}[definition]{\hei 命题}
%\newtheorem{lemma}[definition]{\hei 引理}
%\newtheorem{exercise}[definition]{}
%\newtheorem{corollary}[definition]{\hei 推论}
%\newtheorem{remark}[definition]{\hei 注解}
%%%%%%%%%%%%%%%%%%%%%%%%%%%%%%%%%%%%%%%%%%%%%%%%%%%%%%%%%%%%%%%%%%%
%解决原proof定理环境的两个问题:
%  1. proof 中的item缩进不对
%  2. proof 中的最后一个公式下出现一个黑方块。
%\theoremsymbol{$\blacksquare$}
%\newtheorem{proof}{\hei 证明}
\newenvironment{proof}{\noindent{\hei 证明:}}{\hfill $ \square $ \vskip 4mm}
\theoremsymbol{$\square$}


%%%%%%%%%%%%%%%%%%%%%%%%%%%%%%%%%%%%%%%%%%%%%%%%%%%%%%%%%%%
% 用于中文段落缩进 和正文版式
%%%%%%%%%%%%%%%%%%%%%%%%%%%%%%%%%%%%%%%%%%%%%%%%%%%%%%%%%%%

\ifx\atempxetex\usewhat
\newcommand{\CJKcaption}[1]{
  \ifx\CJK@actualBinding \@empty
    \PackageError{CJK}{
      You must be inside of a CJK environment to use \protect\CJKcaption}{}
  \else
    \makeatletter
    \InputIfFileExists{#1.cpx}{}{
      \PackageError{CJK}{
        Can't find #1.cpx}{
        The default captions are used if you continue.}}
    \makeatother
  \fi}
\CJKcaption{gb_452}
\else
\CJKcaption{gb_452}
\newlength \CJKtwospaces
\def\CJKindent{
    \settowidth\CJKtwospaces{\CJKchar{"0A1}{"0A1}\CJKchar{"0A1}{"0A1}}%
    \parindent\CJKtwospaces
}
\CJKtilde  \CJKindent
\fi
\renewcommand\CJKprechaptername{}
\renewcommand\CJKchaptername{}
\setlength{\parindent}{26pt} %由于工大论文的每行的字距加大了,需要增加段首缩2pt

\renewcommand\contentsname{\hei 目~~~~录}

%%%%%%章节标题为“1”的形式
\renewcommand\chaptername{\arabic{chapter}}
%%%%%%%%%%%%%%%%%%%%%%%%%%%%%%%%%%%%%%%%%%%%%%%%%%
%定义段落章节的标题和目录项的格式
%%%%%%%%%%%%%%%%%%%%%%%%%%%%%%%%%%%%%%%%%%%%%%%%%%
\setcounter{secnumdepth}{4} \setcounter{tocdepth}{2}

\titleformat{\chapter}[hang]{\xiaosan\bf\filcenter\song\sf\boldmath}{\xiaosan\chaptertitlename}{1em}{\xiaosan}
\titlespacing{\chapter}{0pt}{18pt}{18pt}

\titleformat{\section}[hang]{\song\sihao\sf\bf\boldmath}{\song\sihao\sf\bf\thesection}{0.5em}{}
\titlespacing{\section}{0pt}{12pt}{12pt}

\titleformat{\subsection}[hang]{\song\sf\xiaosi\bf\boldmath}{\xiaosi\thesubsection}{0.5em}{}
\titlespacing{\subsection}{0pt}{6pt}{6pt}

\titleformat{\subsubsection}[hang]{\song\sf\xiaosi\boldmath}{\xiaosi\thesubsubsection}{0.5em}{}[\;\;]
\titlespacing{\subsubsection}{0pt}{3pt}{2pt}

% 缩小目录中各级标题之间的缩进,使它们相隔一个字符距离,也就是12pt
\makeatletter
\renewcommand*\l@section{\@dottedtocline{1}{12pt}{18pt}}
\renewcommand*\l@subsection{\@dottedtocline{2}{24pt}{27pt}}
\renewcommand*\l@subsubsection{\@dottedtocline{3}{36pt}{39pt}}
\renewcommand*\l@paragraph{\@dottedtocline{4}{48pt}{48pt}}
\renewcommand*\l@subparagraph{\@dottedtocline{5}{60pt}{60pt}}

%控制中文目录
\dottedcontents{chapter}[3.4em]{\vspace{0.5em}\hspace{-3.4em}\hei \bf\boldmath}{0.0em}{5pt}% 章标题后用粗点
%\titlecontents{chapter}[3.92em]{\vspace{0.5em}\hspace{-3.92em}\hei \bf\boldmath}{\contentslabel{0em}}{\hspace*{-0em}}{\normalfont\titlerule*[5pt]{.}\contentspage} %章标题后用细点
\dottedcontents{section}[1.6cm]{}{1.8em}{5pt}
\dottedcontents{subsection}[2.840cm]{}{2.7em}{5pt}
\dottedcontents{subsubsection}[3.78cm]{}{3.4em}{5pt}

%%%%%%%%%%%%%%%%%%%%%%%%%%%%%%%%%%%%%%%%%%%%%%%%%%%%%%%
% 定义页眉和页脚 使用fancyhdr 宏包
%%%%%%%%%%%%%%%%%%%%%%%%%%%%%%%%%%%%%%%%%%%%%%%%%%%%%%%%
\newcommand{\makeheadrule}{%
\makebox[-3pt][l]{\rule[.7\baselineskip]{\headwidth}{0.4pt}}
\rule[0.85\baselineskip]{\headwidth}{0pt}\vskip-.8\baselineskip}%控制页眉上的线粗细
\renewcommand{\headrule}{%
    {\if@fancyplain\let\headrulewidth\plainheadrulewidth\fi
     \makeheadrule}}

\pagestyle{fancyplain}

%去掉章节标题中的数字
%%不要注销这一行,否则页眉会变成:“第1章1  绪论”样式
\renewcommand{\chaptermark}[1]{\markboth{\chaptertitlename~~ \ #1}{}}
 \fancyhf{}

%在book文件类别下,\leftmark自动存录各章之章名,\rightmark记录节标题
%% 页眉字号 南理工要求小五宋体
%根据单双面打印设置不同的页眉;
\ifoneortwoside
  \fancyhead{} % clear all fields
  \fancyhead[LO]{\CJKfamily{song}\xiaowu{}\cxuewei 论文}
  \fancyhead[RO]{\CJKfamily{song}\xiaowu 基于图的模式识别及其在计算机视觉中的应用}
  \fancyhead[LE]{\CJKfamily{song}\xiaowu \leftmark}%
  \fancyhead[RE]{\CJKfamily{song}\xiaowu  \cxuewei 论文}%
  \fancyfoot[RO,LE]{\xiaowu ~\thepage~} %页码左右两边有一小段横线%\if@mainmatter \fi
\else
\fancyhead{} % clear all fields
  \fancyhead[LO]{\CJKfamily{song}\xiaowu{}\cxuewei 论文}
  \fancyhead[RO]{\CJKfamily{song}\xiaowu{}基于图的模式识别及其在计算机视觉中的应用}
  \fancyhead[LE]{\CJKfamily{song}\xiaowu\leftmark }%
  \fancyhead[RE]{\CJKfamily{song}\xiaowu  \cxuewei 论文 }%
  \fancyfoot[RO,LE]{\xiaowu ~\thepage~} %\if@mainmatter \fi
\fi

\renewcommand\frontmatter{%
    \cleardoublepage
  \@mainmatterfalse
  \pagenumbering{Roman}}
%%%%%%%%%%%%%%%%%%%%%%%%%%%%%%%%%%%%%%%%%%%%%%%%%%%%%%%%
% 设置行距和段落间垂直距离
%%%%%%%%%%%%%%%%%%%%%%%%%%%%%%%%%%%%%%%%%%%%%%%%%%%%%%%%
\renewcommand{\CJKglue}{\hskip 0.3pt plus 0.08\baselineskip}%加大字间距,使每行33个字
%\setlength{\belowcaptionskip}{10pt}   % 加大标题和表格之间的距离 \abovecaptionskip 默认是10pt
\setlength{\parskip}{3pt plus1pt minus1pt} % 段落之间的竖直距离
\renewcommand{\baselinestretch}{2}% 定义行距
%%%%%%%%%%%%%%%%%%%%%%%%%%%%%%%%%%%%%%%%%%%%%%%%%%%%%%%%
% 调整列表环境的垂直间距
%%%%%%%%%%%%%%%%%%%%%%%%%%%%%%%%%%%%%%%%%%%%%%%%%%%%%%%%
\setitemize{itemindent=38pt,leftmargin=0pt,itemsep=-0.4ex,listparindent=26pt,partopsep=0pt,parsep=0.5ex,topsep=-0.25ex}
\setenumerate{itemindent=38pt,leftmargin=0pt,itemsep=-0.4ex,listparindent=26pt,partopsep=0pt,parsep=0.5ex,topsep=-0.25ex}
\setdescription{itemindent=38pt,leftmargin=0pt,itemsep=-0.4ex,listparindent=26pt,partopsep=0pt,parsep=0.5ex,topsep=-0.25ex}

\newcommand{\ucite}[1]{$^{\mbox{\scriptsize \cite{#1}}}$} % 增加 \ucite 命令使显示的引用为上标形式
\newcommand{\citeup}[1]{$^{\mbox{\scriptsize \cite{#1}}}$} % for WinEdt users

%%%%%%%%%%%%%%%%%%%%%%%%%%%%%%%%%%%%%%%%%%%%%%%%%%%%%%%%%%%
% 定制浮动图形和表格标题样式 %这里用ccaption完全代替了caption2的功能
\captionstyle{\centering}   %不同的图标题形式采用不同的命令
%\indentcaption{0pt}           %参见ccaption
\hangcaption
\captionnamefont{\CJKfamily{song}\rmfamily\wuhao\selectfont}
\captiontitlefont{\CJKfamily{song}\rmfamily\wuhao\selectfont}
\captiondelim{~} %~

%%%%%%%%%%%%%%%%%%%%%%%%%%%%%%%%%%%%%%%%%%%%%%%%%%%%%%%
% 定义题头格言的格式
% 用法 \begin{Aphorism}{author}
%         aphorism
%      \end{Aphorism}
\newsavebox{\AphorismAuthor}
\newenvironment{Aphorism}[1]
{\vspace{0.5cm}\begin{sloppypar} \slshape
\sbox{\AphorismAuthor}{#1}
\begin{quote}\small\itshape }
{\\ \hspace*{\fill}------\hspace{0.2cm} \usebox{\AphorismAuthor}
\end{quote}
\end{sloppypar}\vspace{0.5cm}}

%自定义一个空命令,用于注释掉文本中不需要的部分。
\newcommand{\comment}[1]{}

\renewcommand\contentsname{\hei 目~~~~录}
\renewcommand\listfigurename{\hei 插图目录}
\renewcommand\listtablename{\hei 表格目录}

%%%%%%将章标题中的中文数字(一、二、三)变为阿拉伯数字(1,2,3)
\renewcommand\CJKthechapter{%\CJKnumber
{\@arabic\c@chapter}}

%%%%%%不要拉大行距使得页面充满
\raggedbottom

% This is the flag for longer version
\newcommand{\longer}[2]{#1}

\newcommand{\ds}{\displaystyle}

% define graph scale
\def\gs{1.0}

%%%%%%%%%%%%%%%%%%%%%%%%%%%%%%%%%%%%%%%%%%%%%%%%%%%%%%%%%%%%%%%%%%%%%%
% 自定义项目列表标签及格式 \begin{hitlist} 列表项 \end{hitlist}
%%%%%%%%%%%%%%%%%%%%%%%%%%%%%%%%%%%%%%%%%%%%%%%%%%%%%%%%%%%%%%%%%%%%%%
\newcounter{hitctr} %自定义新计数器
\newenvironment{hitlist}{%%%%%定义新环境
\begin{list}{{\hei (\arabic{hitctr})}} %%标签格式
    {
     \usecounter{hitctr}
     \setlength{\leftmargin}{0cm}     %左边界
     \setlength{\parsep}{0ex}         %段落间距
     \setlength{\topsep}{0pt}         %列表到上下文的垂直距离
     \setlength{\itemsep}{0ex}        %标签间距
     \setlength{\labelsep}{0.3em}     %标号和列表项之间的距离,默认0.5em
     \setlength{\itemindent}{46pt}    %标签缩进量
     \setlength{\listparindent}{27pt} %段落缩进量
    }}
{\end{list}}%%%%%

%%%%%%%%%%%%%%%%%%%%%%%%%%%%%%%%%%%%%%%%%%%%%%%%%%%%%%%%%%%%%%%%%%%%%%
% 自定义项目列表标签及格式 \begin{publist} 列表项 \end{publist}
%%%%%%%%%%%%%%%%%%%%%%%%%%%%%%%%%%%%%%%%%%%%%%%%%%%%%%%%%%%%%%%%%%%%%%
\newcounter{pubctr} %自定义新计数器
\newenvironment{publist}{%%%%%定义新环境
\begin{list}{\arabic{pubctr}} %%标签格式
    {
     \usecounter{pubctr}
     \setlength{\leftmargin}{2em}     % 左边界 \leftmargin =\itemindent + \labelwidth + \labelsep
     \setlength{\itemindent}{0em}     % 标号缩进量
     \setlength{\labelwidth}{1em}     % 标号宽度
     \setlength{\labelsep}{1em}       % 标号和列表项之间的距离,默认0.5em
     \setlength{\rightmargin}{0em}    % 右边界
     \setlength{\topsep}{0ex}         % 列表到上下文的垂直距离
%     \setlength{\partopsep}{0ex}      % 列表是一个新的段落时,加的额外到上下文的距离
     \setlength{\parsep}{0ex}         % 段落间距
     \setlength{\itemsep}{0ex}        % 标签间距
     \setlength{\listparindent}{26pt} % 段落缩进量
    }}
{\end{list}}%%%%%

%%%%%%%%%%%%%%%%%%%%%%%%%%%%%%%%%%%%%%%%%%%%%%%%%%%%%%%%%%%%%%%%%%%%%%
% 默认字体
\renewcommand\normalsize{%
  \@setfontsize\normalsize{12.1pt}{13pt}
  \setlength\abovedisplayskip{8pt plus 2pt minus 2pt}
  \setlength\abovedisplayshortskip{7pt plus 2pt minus 2pt}
  \setlength\belowdisplayskip{\abovedisplayskip}
  \setlength\belowdisplayshortskip{\abovedisplayshortskip}
  \setlength\jot{6pt}
  \let\@listi\@listI}
\def\defaultfont{\renewcommand{\baselinestretch}{1.37}\normalsize\selectfont}
\predisplaypenalty=0  %公式之前可以换页,公式出现在页面顶部
%%%%%%%%%%%%%%%%%%%%%%%%%%%%%%%%%%%%%%%%%%%%%%%%%%%%%%%%%%
% 定义可以指定宽度的下划线
\def\NJUSTunderline[#1]#2{
\underline{\hbox to #1{\hfill#2\hfill}}}\def\@NJUSTunderline{}
%%%%%%%%%%%%%%%%%%%%%%%%%%%%%%%%%%%%%%%%%%%%%%%%%%%%%%%%%%%%%%%%%%%%%%
% 封面、摘要、版权、致谢格式定义
%--------定义标签对应选项的值-----------------------------
\def\natclassifiedindex#1{\def\@natclassifiedindex{#1}}\def\@natclassifiedindex{}
\def\secretclassifiedindex#1{\def\@secretclassifiedindex{#1}}\def\@secretclassifiedindex{}%密级
\def\internatclassifiedindex#1{\def\@internatclassifiedindex{#1}}\def\@internatclassifiedindex{}

\def\cname#1{\def\@cname{#1}}\def\@cname{}
\def\crole#1{\def\@crole{#1}}\def\@crole{}
\def\school#1{\def\@school{#1}}\def\@school{}
\def\ctitle#1{\def\@ctitle{#1}}\def\@ctitle{}
\def\cdegree#1{\def\@cdegree{#1}}\def\@cdegree{}
\def\csubject#1{\def\@csubject{#1}}\def\@csubject{}
\def\cauthor#1{\def\@cauthor{#1}}\def\@cauthor{}
\def\csupervisor#1{\def\@csupervisor{#1}}\def\@csupervisor{}
%\def\cassosupervisor#1{\def\@cassosupervisor{~ & {\hei 副 \hfill 导 \hfill 师:} & #1\\}}\def\@cassosupervisor{}
%\def\ccosupervisor#1{\def\@ccosupervisor{~ & {\hei 联 \hfill 合\hfill 导 \hfill 师:} & #1\\}}\def\@ccosupervisor{}
\def\cdate#1{\def\@cdate{#1}}\def\@cdate{}
\def\ddate#1{\def\@ddate{#1}}\def\@ddate{}%定义答辩日期
\def\chairman#1{\def\@chairman{#1}}\def\@chairman{}
\def\expositor#1{\def\@expositor{#1}}\def\@expositor{}


%--------------定义标签(label)---------------------
\def\lnatclassifiedindex#1{\def\@lnatclassifiedindex{#1}}\def\@lnatclassifiedindex{}
\def\lsecretclassifiedindex#1{\def\@lsecretclassifiedindex{#1}}\def\@lsecretclassifiedindex{}%密级
\def\linternatclassifiedindex#1{\def\@linternatclassifiedindex{#1}}\def\@linternatclassifiedindex{}

\def\lname#1{\def\@lname{#1}}\def\@lname{}
\def\lauthor#1{\def\@lauthor{#1}}\def\@lauthor{}
\def\lschool#1{\def\@lschool{#1}}\def\@lschool{}
\def\ldegree#1{\def\@ldegree{#1}}\def\@ldegree{}
\def\lsubject#1{\def\@lsubject{#1}}\def\@lsubject{}
\def\lsupervisor#1{\def\@lsupervisor{#1}}\def\@lsupervisor{}
%\def\lassosupervisor#1{\def\@lassosupervisor{~ & {\hei 副 \hfill 导 \hfill 师:} & #1\\}}\def\@lassosupervisor{}
%\def\lcosupervisor#1{\def\@lcosupervisor{~ & {\hei 联 \hfill 合\hfill 导 \hfill 师:} & #1\\}}\def\@lcosupervisor{}
\def\ldate#1{\def\@ldate{#1}}\def\@ldate{}
\def\lddate#1{\def\@lddate{#1}}\def\@lddate{}%定义答辩日期
\def\lchairman#1{\def\@lchairman{#1}}\def\@lchairman{}
\def\lexpositor#1{\def\@lexpositor{#1}}\def\@lexpositor{}

\long\def\cabstract#1{\long\def\@cabstract{#1}}\long\def\@cabstract{}%摘要标签
\def\ckeywords#1{\def\@ckeywords{#1}}\def\@ckeywords{}%关键词标签

%--------------定义英文封面---------------------
\def\etitle#1{\def\@etitle{#1}}\def\@etitle{}
\def\edegree#1{\def\@edegree{#1}}\def\@edegree{}
\def\eaffil#1{\def\@eaffil{#1}}\def\@eaffil{}
\def\esubject#1{\def\@esubject{#1}}\def\@esubject{}
\def\eauthor#1{\def\@eauthor{#1}}\def\@eauthor{}
\def\esupervisor#1{\def\@esupervisor{#1}}\def\@esupervisor{}
%\def\eassosupervisor#1{\def\@eassosupervisor{#1}}\def\@eassosupervisor{}
%\def\eassosupervisor#1{\def\@eassosupervisor{~ & \textbf{Associate Supervisor:} & #1\\}}\def\@eassosupervisor{}
\def\ecosupervisor#1{\def\@ecosupervisor{#1}}\def\@ecosupervisor{}
%\def\ecosupervisor#1{\def\@ecosupervisor{~ & \textbf{Co Supervisor:} & #1\\}}\def\@ecosupervisor{}
\def\edate#1{\def\@edate{#1}}\def\@edate{}
\long\def\eabstract#1{\long\def\@eabstract{#1}}\long\def\@eabstract{}
\long\def\NotationList#1{\long\def\@NotationList{#1}}\long\def\@NotationList{}
\def\ekeywords#1{\def\@ekeywords{#1}}\def\@ekeywords{}
%%%%%%%%%%%%%%%%%%%%%%%%%%%%%%%%%%%%%%%%%%%%%%%%%%%%%%%%%%%%%%%
% 定义封面
\def\makecover{
    \normalbiao %表格字号设置
     %封面一
    \newpage
    \thispagestyle{empty}
    \begin{center}

      \parbox[t][0.6cm][t]{\textwidth}{
      \begin{center} \end{center}}

      \parbox[t][2.2cm][t]{\textwidth}{%
       \song \wuhao
       \@lnatclassifiedindex~
       \NJUSTunderline[100pt]{\@natclassifiedindex }
        \hfill
       \@lsecretclassifiedindex~
       \NJUSTunderline[100pt]{\@secretclassifiedindex}
        \vskip 10pt
        \@linternatclassifiedindex~
        \NJUSTunderline[108pt]{\@internatclassifiedindex}\hfill
       }%

      \parbox[t][2.7cm][b]{\textwidth}{\song\yihao
      \begin{center} {\textbf{学\qquad 位\qquad 论\qquad 文}}\end{center} }

      \setlength{\baselineskip}{1.5\baselineskip}
      \parbox[t][3.0cm][b]{\textwidth}{\erhao
      \begin{center} {\hei\textbf{\@school\@cdegree\@ctitle}}\end{center} }

      \begin{center} {\xiaoer\kai{\NJUSTunderline[100pt]{\textbf{\@cauthor}}}} \end{center}%还可设置上下间隔

      \parbox[t][0.8cm][t]{\textwidth}{\begin{center}  \end{center} }

      \parbox[t][6cm][c]{\textwidth}{
      \begin{center}
        \begin{tabular}{llcll}
        \song\xiaosi\@lsupervisor\@lname~&
        \multicolumn{4}{l}{\NJUSTunderline[320pt]{\kai\sanhao\textbf{\@csupervisor}\hspace*{2em}\kai\sihao\@crole}}\vspace{18pt}\\
        \hspace*{2.5cm}&
        \multicolumn{4}{l}{\NJUSTunderline[320pt]{}}\vspace{18pt}\\
        %\NJUSTunderline[115pt]{\@ccosupervisor}
        %\NJUSTunderline[115pt]{\@cassosupervisor}
        \song\xiaosi\@ldegree &
        \NJUSTunderline[100pt]{\kai\sanhao\textbf{\@cdegree}}&\hspace*{1pt}&
        \song\xiaosi\@lsubject~&
        \NJUSTunderline[100pt]{\kai\sanhao\textbf{\@csubject}}\vspace{18pt}\\
        \song\xiaosi\@ldate~&
        \NJUSTunderline[100pt]{\defaultfont\@cdate}&\hspace*{2pt}&
        \song\xiaosi\@lddate~&
        \NJUSTunderline[100pt]{\defaultfont\@ddate}\vspace{18pt}\\
        \multicolumn{5}{l}{\song\xiaosi\@lschool~~\NJUSTunderline[285pt]{\kai\sanhao\@school}}\vspace{18pt}\\
        \multicolumn{4}{r}{\song\xiaosi\@lchairman}&
        \NJUSTunderline[100pt]{\@chairman}\vspace{18pt}\\
        \multicolumn{4}{r}{\song\xiaosi\@lexpositor}&
        \NJUSTunderline[100pt]{\@expositor}\vspace{18pt}
       \end{tabular}
     \end{center}}

   \end{center}
%%%%%%封面一增加一空白页
  \ifoneortwoside
    \newpage
    ~~~\vspace{1em}
    \thispagestyle{empty}
  \fi
%%%%%%%%%内封
    \begin{titlepage}
    \begin{center}
      \parbox[t][1cm][b]{\textwidth}{\xiaoer
      \begin{center} {\kai \ifxueweimaster\cxueke\fi 中华人民共和国\cxueke\cxuewei 学位论文 }\end{center} }

      \parbox[t][0.8cm][t]{\textwidth}{
      \begin{center} \end{center} }

      \parbox[t][2cm][t]{\textwidth}{\xiaoer
      \begin{center} {\hei\textbf{\@school\@cdegree\@ctitle}}\end{center} }

      \ifxueweidoctor
      \parbox[t][3.8cm][t]{\textwidth}{\xiaoer
      \begin{center} {\hei  \@etitle}\end{center} }
      \else
      \parbox[t][3.8cm][t]{\textwidth}{\centering
        \ }
      \fi

    \parbox[t][2.0cm][t]{\textwidth}{\kai\xiaoer
    \begin{center}
    \begin{tabular}{lcc}
    \textbf{\@lauthor:}&
    \multicolumn{2}{l}{\textbf{\@cauthor}}\\
    \textbf{\@lsupervisor:}&
    \textbf{\@csupervisor }&
    \textbf{\@crole}
    \end{tabular}
    \end{center}}

    \ifxueweidoctor
    %\parbox[t][8.5cm][t]{\textwidth}{\centering
    \parbox[c][8.5cm][c]{\textwidth}{\centering
        \includegraphics[width = 7cm,height=7cm]{xiaohui}}
    \else
    \parbox[t][8.5cm][t]{\textwidth}{\centering \ }
    \fi

    \parbox[t][1.2cm][t]{\textwidth}{\xiaoer
    \begin{center} {\song  \textbf{\@school}}  \end{center} }

    \parbox[t][0.5cm][t]{\textwidth}{
    \begin{center} {\song \xiaoer \textbf{\@cdate}} \end{center} }
   \end{center}

%%%% 内封增加一空白页
    \ifoneortwoside
      \newpage
      ~~~\vspace{1em}
      \thispagestyle{empty}
    \fi

%%%%% 英文封面
    \newpage
    \thispagestyle{empty}
    \begin{center}
    \parbox[t][0.6cm][t]{\textwidth}{
    \begin{center} \end{center}}

   \parbox[t][2.2cm][t]{\textwidth}{}

    \parbox[t][1cm][b]{\textwidth}{\xiaoer
    \begin{center} {PH. D. Dissertation}\end{center} }

    \parbox[t][2mm][t]{\textwidth}{
    \begin{center}  \end{center}}

    \parbox[t][3.0cm][b]{\textwidth}{\erhao
    \begin{center} {\textbf{\@etitle}}\end{center} }

    \parbox[t][3.0cm][t]{\textwidth}{
    \begin{center}  \end{center}}

    \parbox[t][2.5mm][b]{\textwidth}{\xiaoer
    \begin{center} { \emph{By}}\end{center}

    \begin{center} { \emph{\textbf{\@eauthor}}}\end{center}}

    \parbox[t][1.2cm][b]{\textwidth}{\xiaoer
    \begin{center} {\emph{Supervised by Prof.~\textbf{\@esupervisor}}}\end{center}}

    \parbox[t][6.0cm][t]{\textwidth}{
    \begin{center}  \end{center} }

    \parbox[t][2.6cm][b]{\textwidth}{\xiaoer
    \begin{center} {Nanjing University of  Science \& Technology}\end{center}

    \begin{center} {\@edate }\end{center}}

   \end{center}
  \end{titlepage}
%%%%%%英文封面增加一空白页
  \ifoneortwoside
    \newpage
    ~~~\vspace{1em}
    \thispagestyle{empty}
  \fi

 \thispagestyle{empty}
 % !Mode:: "TeX:UTF-8"
% 这个是为了WinEdt设置的,它的默认不是UTF8.
% !TeX root = ../main.tex
% 若xelatex编译非UTF8文件,需在每个文件中指定字符编码;
% main.tex中手动制定了\atemp和\usewhat参数;
\ifx\atempxetex\usewhat
%\XeTeXinputencoding "gbk"
\fi
\defaultfont

\thispagestyle{empty}
%在此节中,由于用了与其它正文不同的字体,此节中用的是
%四号字体,所以要对行距与首行缩进重新定义。
\begin{center}{\sanhao \hei{声\quad 明}}\end{center}

\renewcommand{\baselinestretch}{1.5}\large{}
{\setlength{\parindent}{2em}本学位论文是我在导师的指导下取得的研究成果,尽我所知,在本
 学位论文中,除了加以标注和致谢的部分外,不包含其他人已经发表或
公布过的研究成果,也不包含我为获得任何教育机构的学位或学历而使
用过的材料。与我一同工作的同事对本学位论文做出的贡献均已在论文
中作了明确的说明。}

    \vspace{0.738cm}
    \begin{flushleft}{
    研究生签名:\underline{~~~~~~~~~~~~~~~}~~~~~~~~~~~~~~~~日期:~~~~~~~~~~~年~~~~~月~~~~~日}
    \end{flushleft}

    \vspace{2.214cm}
%%%%%%%%%%%%%%%%%%南京理工大学博(硕)士学位论文使用授权书%%%%%%%%%%%%%%%%%%%
%\phantomsection
    \begin{center}{\sanhao \hei{南京理工大学\cxuewei 学位论文使用授权声明}}
    \end{center}

    \vspace{0.738cm}

\renewcommand{\baselinestretch}{1.5}\large{}
{\setlength{\parindent}{2em}南京理工大学有权保存本学位论文的电子和纸质文档,可以借阅或上网公布本学位论文的部分或全部内容,
可以向有关部门或机构送交并授权其保存、借阅或上网公布本学位论文的部分或全部内容。对于保密论文,
按保密的有关规定和程序处理。}


\vspace{1.476cm}
\begin{flushleft}{
研究生签名:\underline{~~~~~~~~~~~~~~~~~}~~~~~~~~~~~~~~日期:~~~~~~~~~~~年~~~~~月~~~~~日}
\end{flushleft}
\vspace{0.2cm}
\iffalse
\begin{flushleft}{
\hspace*{8pt}导师签名:\underline{~~~~~~~~~~~~~~~~~~~}~~~~~~~~~~~~~~日期:~~~~~~~~~~~年~~~~~月~~~~~日}
\end{flushleft}
\fi
\newpage
   % 原创性申明,即让原创性申明在摘要前显示

%%%%%%增加一空白页
 \ifoneortwoside
    \newpage
    ~~~\vspace{1em}
    \thispagestyle{empty}
  \fi
%%%%%%%%%%%%%%%%%%%   Abstract and keywords  %%%%%%%%%%%%%%%%%%%%%%%修改摘要和关键词???????????
\clearpage \BiAppendixChapter{\textbf{摘\hspace*{2em}要}}{Abstract (in Chinese)} %不要挪到下一行,生成正确的摘要toc
\setcounter{page}{1}
\song \normalsize
\defaultfont
\@cabstract \vspace{1em} \hangafter1\hangindent4.28em\noindent
{\sihao\song \textbf{关键词}} \quad \@ckeywords

%%%%%%%%%%%%%%%%%%%   English Abstract  %%%%%%%%%%%%%%%%%%%%%%%%%%%%%%
\clearpage
\defaultfont \BiAppendixChapter{\textbf{Abstract}}{Abstract (in English)} %不要挪到下一行,生成正确的摘要toc
\@eabstract
\vspace{1em}

\hangafter1\hangindent5.5em\noindent
{\textbf{Keywords}} \quad \@ekeywords
\wuhaobiao  %正文表格设置
}

%%%%%%%%%%%%%%%%%%%%%%%%%%%%%%%%%%%%%%%%%%%%%%%%%%%%%%%%%%%%%%%

\urlstyle{same}  %论文中引用的网址的字体默认与正文中字体不一致,这里修正为一致的。

%主要符号表 \NotationList
\long\def\notation{ \clearpage
\BiAppendixChapter{\textbf{主要符号说明}}{} \normalbiao
\@NotationList \wuhaobiao}

%%% 五号字表格设置 start %%%%%
\gdef\tpltable{\relax}
\let\tpltable\longtable
\gdef\wuhaobiao{%五号字
    \def\tabular{\wuhao\gdef\@halignto{}\@tabular}
    \def\endtabular{\endarray $\egroup \defaultfont}
    \def\longtable{\wuhao\tpltable}
    \def\endlongtable{\adl@LTlastrow \adl@org@endlongtable\defaultfont}
}
\gdef\normalbiao{%正常字号
    \def\tabular{\gdef\@halignto{}\@tabular}
    \def\endtabular{\endarray $\egroup}
    \def\longtable{\tpltable}
    \def\endlongtable{\adl@LTlastrow \adl@org@endlongtable}
}
\wuhaobiao
%%% 五号字表格设置 end %%%%%
%\renewcommand{\arraystretch}{1.4} %表格中行距 ,导致公式 bmatrix 间距增大。

% 表格与下方间距
\renewcommand\endtable{\vspace{-2mm}\end@float}
% 算法与下方间距
\renewcommand\endalgorithm{\@algocf@finish \ifthenelse {\equal {\algocf@float }{figure}}{\end {figure}}{
\@algocf@term@caption \ifthenelse {\boolean {algocf@algoH}}{\end {algocf@Here}}
{\end {algocf}}}\@algocf@term\vspace{-5mm}}

\makeatother
